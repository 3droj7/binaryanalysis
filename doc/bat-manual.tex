\documentclass[10pt]{article}
\usepackage{url}
\usepackage{a4}

\pagestyle{empty}

\setlength{\parskip}{4pt}

\author{Armijn Hemel -- Tjaldur Software Governance Solutions}
\title{Binary Analysis Tool User and Developer Manual - describing version 19}

\begin{document}

\maketitle
\thispagestyle{empty}

\tableofcontents

\section{Introducing the Binary Analysis Tool}

The Binary Analysis Tool (BAT) is a framework that can help developers and
companies check binary files. Its primary application is for Open Source
software license compliance, with a special focus on supply chain management
in consumer electronics.

BAT consists of several programs written in Python. The most important program
is the scanner for binary objects to unpack binaries recursively and apply a
anumber of scans, for example for open source license compliance, visualising
linking information, finding version information, and so on.  There are also
other programs to help with specific compliance tasks, such as verifying if
configurations for a given BusyBox binary match with the configuration in
source code. Also included is a very experimental program to derive a possible
configuration from a Linux kernel image, as well as programs to verify results
from a binary scan with a source code archive.

\section{Installing the Binary Analysis Tool}

\subsection{Hardware requirements}

The tools in the Binary Analysis Tool can be quite resource intensive. These
are largely I/O-bound (database access), so it better to invest in faster disks
or ramdisks than in raw CPU power. Using more cores is also highly recommended,
since most of the programs in the Binary Analysis Tool will vastly benefit from
this.

\subsection{Software requirements}

To run BAT a recent Linux distribution is needed. Development is (currently)
done on Fedora 19 and 20 and Ubuntu 14.04, so those platforms are likely to
work best. Other platforms (like Debian 7) are tested with, but only to
build binary releases. Ubuntu versions older than 14.04, will not work due to
a broken version of the PyDot package.

If the latest version from version control is used it is important to look at
the file \texttt{setup.cfg} to get a list of the dependencies that should be
met on the host system before installing BAT if the host system is Fedora. If
the host system is Ubuntu or Debian this information will be in
\texttt{debian/control}.

\subsubsection{Installation on Fedora}

To install on Fedora three packages are needed:
\texttt{bat-extratools}, \texttt{bat-extratools-java} and \texttt{bat}. These
can be downloaded from the BAT website in both prebuilt versions and as source
RPM files. When installing the three files there should be a list of
dependencies that should be installed to let BAT work successfully. Some of
the dependencies are not in Fedora by default but need to be installed through
external repositories, such as RPMfusion.

\subsubsection{Installation on Debian and Ubuntu}

To install on Debian and Ubuntu three packages are needed:
\texttt{bat-extratools}, \texttt{bat-extratools-java} and \texttt{bat}. These
can be downloaded from the BAT website as binary DEB files. When installing the
three files there should be a list of dependencies that should be installed to
let BAT work successfully. Some of these packages are not in Debian by default
but need to be installed by enabling extra repositories such as Debian
\texttt{non-free}.

The graphical interface might not work on Debian 6, because of a too old
version of \texttt{wxPython} that is shipped in Debian 6.

\section{Analysing binaries with the Binary Analysis Tool}

BAT consists of several programs and a few helper scripts (not meant to be
used directly). The main purpose of the Binary Analysis tool is to analyse
arbitrary binaries and review results. Analysis of the binary is done via
a commandline tool (\texttt{bat-scan}), while the results can be viewed using
a special graphical viewer (\texttt{batgui}).

\subsection{Running \texttt{bat-scan}}

Invoking \texttt{bat-scan} is simple. You will need to supply three
parameters:

\begin{enumerate}
\item \texttt{-c} : path to a configuration file
\item \texttt{-b} : path to the binary to be scanned
\item \texttt{-o} : path to an output file, where unpacked files, reports,
plus the final program state be written to. This file can later be opened with
the viewer.
\end{enumerate}

The default install of BAT comes with a configuration file (installed in
\texttt{/etc/bat/} although this will likely change in the future) with default
settings that have proven to work well but almost everything can be changed or
tweaked. A lengthy explanation of the different types of scans and their
configuration can be found in the appendix.

A typical invocation looks like this:

\texttt{python bat-scan -c /path/to/configuration -b /path/to/binary -o
/path/to/outputfile}

XML output, if enabled in the configuration file, will be written to standard
output. Any debugging messages or error messages will appear on standard error.

\subsection{Interpreting the results}

There are two formats in which \texttt{bat-scan} can output its results:

\begin{enumerate}
\item dump file containing program state, complete unpacked directory tree
containing all unpacked data (unless \texttt{outputlite} was set to
\texttt{yes}), plus possibly some extra generated data, such as pictures and
more reporting. These dumps are meant to be used by \texttt{batgui}. This is
the preferred format.
\item XML file (optional/configurable)
\end{enumerate}

\subsubsection{XML output}

If configured \texttt{bat-scan} outputs its results in XML format on
standard output. After redirecting the output to a file it is possible to look
at this file with a commandline tool such as \texttt{xml\_pp} or a webbrowser
such as Mozilla Firefox.

This XML file is not meant for human consumption, but for use by for example
reporting tools. A word of warning is needed: the XML format is not very well
designed and not well maintained. If you want to programmatically process
information from BAT it is recommended to use the Python pickles to extract
information instead.

The default XML pretty printer as shipped by BAT outputs a file that starts
with metadata, such as:

\begin{itemize}
\item date, plus time of the scan (local time of the computer, in UTC)
\item name of the file
\item SHA256 cryptographic checksum of the file, uniquely identifying it
\item size of the file
\item filetype as determined by \texttt{file} on a Linux system
\item relative path inside the unpacked system, plus the absolute path inside
the file system, which is useful for later analysis
\end{itemize}

If any of the scans were successful the results of the these scans can
be found in the element \texttt{scans}.

For each successful unpack action the following attributes are reported:

\begin{itemize}
\item name of the scan (corresponding to the name of the scan in the
configuration file)
\item offset in the parent file of the compressed file, file system or media
file
\end{itemize}

\subsubsection{Viewing results with \texttt{batgui}}

The \texttt{batgui} program was made to view the results of the analysis
process easily, without having to dig through XML. The viewer has two modes:
simple and advanced. In simple mode a tree of the unpack results will be shown,
and each file in the tree can be clicked to display more information. Depending
on which scans were run the tree will be decorated with more information, such
as the type of the file (based on tags), or if matches were found with the
ranking method. Using a filtering system (available from the menu) files that
are typically uninteresting for license compliance engineering (empty
files/directories, symbolic links, graphics files and so on) can be ignored.

Information that is shown per file depends on the scans that were run and the
type of file. For most files information like size, type, path (both relative
inside the unpacked binary, as well as absolute in the scanning tree) will be
shown. If the ranking method was enabled results of the ranking process such as
matched strings, function names, a license guess etecetera will be displayed as
well.

In the optional advanced mode more results will be shown, such as a graphical
representation of a file, where every bit in the binary has been assigned a
grayscale value, plus a textual representation of a file generated with
\texttt{hexdump}. Advanced mode is disabled by default, since loading the
additional pictures and data is quite resource intensive and it will only be
useful in very specific cases. It also requires that these special files are
generated by BAT when scanning a file. This is not done by default but needs
an explicit configuration change. Advanced mode might be removed from the GUI
in future versions of BAT.

\section{Additional programs in the Binary Analysis Tool}

\subsection{\texttt{busybox.py} and \texttt{busybox-compare-configs.py}}

Two other tools in BAT are \texttt{busybox-compare-configs.py} and
\texttt{busybox.py} (in the subdirectory \texttt{bat}). These two tools are
specifically used to analyse BusyBox binaries. BusyBox is in widespread use on
embedded devices and the license violations of BusyBox are actively enforced
in court.

BusyBox binaries on embedded machines often have different configurations,
depending on the needs of the manufacturer. Since providing the correct
configuration is one of the requirements for license compliance it is important
to be able to determine the configuration of a BusyBox binary and verify that
there is a corresponding configuration file in the source code release.

The BusyBox processing tools in BAT try to extract the most likely
configuration from the binary and print it in the right format for that version
of BusyBox.

\texttt{busybox.py} is used to extract the configuration from a binary.
Afterwards \texttt{busybox-compare-configs.py} can be used to compare the
extracted configuration with a vendor supplied configuration.

\subsubsection{Extracting a configuration from BusyBox}

Extracting a configuration from a BusyBox executable is done using
\texttt{busybox.py} which can be found in the \texttt{bat} directory. It needs
two commandline parameters: the path to the binary and the path to a directory
containing a directory \texttt{configs} which has files containing mappings
from BusyBox applet names to BusyBox configuration directives. By default this
value is hardcoded as \texttt{/etc/bat}, but this might change in the future.

Output (a configuration) is written to standard output.

\begin{verbatim}
python bat/busybox.py -b /path/to/busybox/binary -c
     /path/to/pre/extracted/configs > /path/to/saved/config
\end{verbatim}

This command will save the configuration to a file, which can be used as an
input to \texttt{busybox-compare-configs.py}.

\subsubsection{Comparing two BusyBox configurations}

After extracting the configuration the extracted configuration can be compared
to another configuration, for example a configuration as supplied by a vendor
in a source code archive:

\begin{verbatim}
python busybox-compare-configs.py -e /path/to/saved/config
    -f /path/to/vendor/configuration -n $version
\end{verbatim}

\subsection{\texttt{comparebinaries.py}}

The \texttt{comparebinaries.py} program compares two file trees with for example
unpacked firmwares. It is intended to find out which differences there are
between two binaries (like firmwares) unpacked with BAT.

There are two scenarios where this program can be used:

\begin{enumerate}
\item comparing an old firmware (that is already known and which has been
verified) to a new firmware (update) and see if there are any differences.
\item comparing a firmware to a rebuild of a firmware as part of compliance
engineering.
\end{enumerate}

A few assumptions are made:

\begin{enumerate}
\item both firmwares were unpacked using the Binary Analysis Tool
\item files that are in the original firmware, but not in the new firmware, are
not reported (example: removed binaries). This will change in a future version.
\item files that are in the new firmware but not not in the original firmware are
reported, since this would mean additions to the firmware which need to be
checked.
\item files that appear in both firmwares but which are not identical are
checked using \texttt{bsdiff} to determine the size of the difference.
\end{enumerate}

With checksums it is easy to find the files that are different. Using
\texttt{bsdiff} it becomes easier to prioritise based on the size of the
difference.

Small differences are probably not very interesting at all:

\begin{enumerate}
\item time stamps (BusyBox, Linux kernel, and others record a time stamp in the
binary)
\item slightly different build system settings (home directories, paths, and
so on).
\end{enumerate}

Bigger differences are of course much more interesting.

\subsection{\texttt{sourcewalk.py}}

This program can quickly determine whether or not source code files in a
directory can be found in known upstream sources. It uses a pregenerated
database containing names and checksums of files (for example the Linux kernel)
and reports whether or not the source code files can be found in the database
based on these checksums.

The purpose of this script is to find source code files that cannot be found in
upstream sources to reduce the search space during a source code audit.

This script will not catch:

\begin{itemize}
\item binary files
\item patch/diff files
\item anything that does not have an extension from the list in the script
\item configuration files/build scripts
\end{itemize}

\subsection{\texttt{verifysourcearchive.py}}

The \texttt{verifysourcearchive.py} program is to verify a source code archive
using the result of a scan done with BAT.

\section{Binary Analysis Tool extratools collection}

To help with unpacking non-standard file systems, or standard file systems for
which there are no tools readily available on Fedora or Ubuntu there is also
a collection of tools that can be used by BAT to unpack more file systems.
These tools are not part of the standard distribution, but have to be installed
separately. They are governed by different license conditions than the core BAT
distribution.

Currently the collection consists of:

\begin{itemize}
\item \texttt{bat-minix} has a Python script to unpack Minix v1 file systems
that are frequently found on older embedded Linux systems, such as IP cameras.
\item modified version of \texttt{code2html} (which is unmaintained by the
upstream author) that adds support for various more programming languages.
\item unmodified version of \texttt{simg2img} needed for converting Android
sparse files to ext4 file system images.
\item unmodified version of \texttt{romfsck} needed for unpacking romfs file
systems.
\item modified version of \texttt{cramfsck} that enables unpacking cramfs
file systems.
\item unmodified version of \texttt{unyaffs} that enables unpacking for some
(but not all) YAFFS2 file systems.
\item various versions of \texttt{unsquashfs} that enable unpacking variants
of SquashFS. These versions have either been lifted from vendor SDKs, the
OpenWrt project, or upstream SquashFS project.
\item \texttt{bat-visualisation} containing a few custom tools to help generate
pictures. These might be removed in the future.
\item two Java projects: \texttt{jdeserialize} and \texttt{ddex}, to help
respectively with unpacking serialized Java files and scanning binary files
from the Dalvik VM (Android).
\end{itemize}

The collection is split in two packages: \texttt{bat-extratools-java} contains
the two Java packages, the \texttt{bat-extratools} package contains the rest.

\appendix

\section{BAT configuration}

\subsection{Inner workings of the analysis process}

BAT uses a brute force approach for analysing a binary. It assumes no prior
knowledge of how a binary is constructed or what is inside the binary. Instead
it tries to determine what is inside by applying a wide range of methods, such
as looking for known identifiers of file systems and compressed files and
running external tools to find contents in the binary.

During scanning of a file the following steps are taken:

\begin{enumerate}
\item identifier search, using a list of known identifiers
\item verifying file type of a file and, if successful, tagging it. Tags can
be used later on to give more information to the scanner.
\item unpacking file systems, compressed files and media files from the file,
possibly carving it out of the file first.
\item repeat steps 1 - 3 for each file that was unpacked in step 3
\item run individual scans on each file if no further unpacking is possible
\item optionally aggregate scan results
\item process results from scans in step 5 and 6
\item pack results into an archive that can be used by the viewer application
\end{enumerate}

\subsubsection{Identifier search}

The first action performed is scanning a file for known identifiers of
compressed files, file systems and media files. The identifers are important
for a few reasons: first, they are used to determine which checks will run. They
are also used frequently throughout the code for verification and speeding up
unpacking. If a scan depends on a specific identifier being present it can be
set using the \texttt{magic} attribute in the configuration. If an identifier
is not defined anywhere in the configuration file as needed it will be skipped
during the identifier search to speed up the identifier search.

The marker search cannot be enabled or disabled via the configuration file.

\subsubsection{Pre-run checks}

Before files are unpacked they are briefly inspected and if possible tagged.
Tags are used to pass hints to methods that are run later to avoid unnecessarily
scanning a file and to reduce the amount of false positives.

For example, files that only contain text are tagged as \texttt{text}, all other
files are tagged as \texttt{binary} (this depends on the implementation of
Python. Python 2 only considers (by default) ASCII to be valid text). Methods
that only work on binaries can then ignore anything that has been tagged as
\texttt{text}.

Other checks that are available are for valid XML, various Android formats,
ELF executables and libraries, graphics files, audio files, and so on.

The prerun checks can easily be identified in the configuration, since it has
its type set to \texttt{prerun}:

\begin{verbatim}
[verifytext]
type        = prerun
module      = bat.prerun
method      = verifyText
priority    = 3
description = Check if file contains just ASCII text
enabled     = yes
\end{verbatim}

Prerun verifiers can optionally make use of tags that are already present by
using \texttt{magic} and \texttt{noscan} attributes, which will be explained
in detail later for the unpackers.

\subsubsection{Unpackers}

Unpackers can be recognized in the configuration because their type is set
to \texttt{unpack}, for example:

\begin{verbatim}
[jffs2]
type        = unpack
module      = bat.fwunpack
method      = searchUnpackJffs2
priority    = 2
magic       = jffs2_le:jffs2_be
noscan      = text:xml:graphics:pdf:compressed:audio:video:mp4:elf:java:resource:dalvik
description = Unpack JFFS2 file systems
enabled     = yes
\end{verbatim}

In BAT 19 the following file systems, compressed files and media files can be
unpacked or extracted:

\begin{itemize}
\item file systems: Android sparse files, cramfs, ext2/ext3/ext4, ISO9660,
JFFS2, Minix (specific variant of v1 often found on older embedded Linux
systems), SquashFS (several variants), romfs, YAFFS2 (specific variants)
\item compressed files and executable formats: 7z, ar, ARJ, BASE64, BZIP2,
compressed Flash, CAB, compress, CPIO, EXE (specific compression methods only),
GZIP, InstallShield (old versions), LRZIP, LZIP, LZMA, LZO, RAR, RPM, RZIP,
serialized Java, TAR, UPX, XZ, ZIP
\item media files: GIF, ICO, PDF, PNG, WOFF
\end{itemize}

Unpacking differs per file type. Most files use one or more identifiers that
can be searched for in a binary blob. Using this information it is possible
to carve out the right parts of a binary blob and verify if it indeed contains
a compressed file, media file or file system.

There is not always an identifier that can be searched for. The YAFFS2 file
system layout for example is dependent on the hardware specifics of the
underlying flash chip. Without knowing these specifics it is not possible to
specifically search for a valid YAFFS2 file system. This scan therefore tries
to run on every file, unless explicitely filtered out (using \texttt{noscan}
and tags).

Other file types (such as ARJ files) have a very generic identifier, so there
are a lot of false positives. This causes a big increase in runtime. The ARJ
unpacker is therefore disabled by default.

LZMA is another special case: there are many different valid headers for LZMA
files, but in practice only a handful are used.

If unpacking is successful a directory with unpacked files is returned, and, if
available, some meta information to avoid duplicate scanning (blacklisting
information and tags). The unpacked files are added to the scan queue and
scanned recursively.

\subsubsection{Leaf scans}

Leaf scans are scans that are run on every single file after unpacking,
including files that contained files that were found and extracted by unpackers.

Leaf scans can be recognized in the configuration because their type is set to
\texttt{leaf}, for example:

\begin{verbatim}
[iptables]
type        = leaf
module      = bat.checks
method      = searchIptables
noscan      = text:xml:graphics:pdf:compressed:audio:video
description = Determine presence of iptables
enabled     = yes
\end{verbatim}

In earlier versions of BAT (16 and earlier) the term \texttt{program} was used
instead of \texttt{leaf}.

The current leaf scans that are available in BAT are:

\begin{itemize}
\item fast string searches: dproxy, ez-ipupdate, iptables, iproute, libusb,
Linux kernel, loadlin, RedBoot, U-Boot, vsftpd, wireless-tools, wpa-supplicant
\item advanced search mode using ranking of strings, function names, variable
names, field names and Java class names using a database (for ELF and Java,
both regular JVM and Dalvik)
\item BusyBox version number
\item dynamic library dependencies (ELF files only)
\item Linux kernel module license (Linux kernel modules only)
\item Linux kernel version number, plus detection for several subsystems
\item PDF meta data extraction
\item architecture
\item presence of URLs indicating an open source license
\item presence of URLs indicating forges/collaborative software development
sites (SourceForge, GitHub, etcetera)
\end{itemize}

The fast string searches are meant for quick sweep scanning only. They have
their limits, can report false positives or not identify a binary. They should
only be used to signal that further inspection is necessary. For a thorough
investigation the advanced search mode should be used. These scans are likely
to be disabled in the future in the default configuration.

\subsubsection{Aggregators}

Sometimes it helps to aggregate results of a number of files, or it could be
useful to perform other actions after all the individual scans have run. The
best example is dealing with JAR-files (Java ARchives). Individual Java class
files often contain too little information to map them reliably to a source
code package.

Typically a class file contains just a few method names, or field names, or
strings. If inner classes are used it can be even worse and information from a
single source code file could be scattered across several class files.

Since Java programs (note: excluding Android) are typically distributed as a
JAR that is either included at runtime or directly executed, similar to an ELF
library or ELF executable, it makes perfect sense to treat the JAR file as a
single unit and aggregate results for the individual class files and assign
them to the JAR file.

Aggregators take all results of the entire scan as input.

Currently the following aggregators are available:

\begin{itemize}
\item aggregating result of individual Java class files in case they come from
the same JAR file.
\item cleaning up/fixing results of duplicate files: often firmwares contain
duplicate files. Sometimes some more information is available to make a better
choice as to which file is the duplicate and which one is the original version
\item checking dynamically linked ELF files
\item finding duplicate files
\item finding licenses and versions of strings and function names that were
found and optionally pruning the result set to remove unlikely results.
\item pruning files from the scan completely if they are not interesting (such
as pictures, or text files) using tags.
\item generating pictures of results of a scan
\item generating reports of results of a scan
\end{itemize}

\subsubsection{Post-run methods}

In BAT there are methods that are run after all the regular work has been
performed, or ``post-run''. These methods should not alter the scan results in
any way, but just use the information from the scanning process. A typical use
case would be to present the data in a nicer to use format than the standard
report, to use more external data sources or generate graphical representations
of data.

The post-run methods have the type \texttt{postrun} in the configuration, for
example:

\begin{verbatim}
[hexdump]
type        = postrun
module      = bat.generatehexdump
method      = generateHexdump
noscan      = text:xml:graphics:pdf:audio:video
envvars     = BAT_REPORTDIR=/tmp/images:BAT_IMAGE_MAXFILESIZE=100000000
description = Create hexdump output of files
enabled     = no
storetarget = reports
storedir    = /tmp/images
storetype   = -hexdump.gz
cleanup     = no
\end{verbatim}

\subsection{Scan configuration}

The analysis process is highly configurable: methods can be simply enabled and
disabled, based on need: some methods can run for quite a long time, which
might be undesirable at times. Configuration is done via a simple configuration
file in Windows INI format.

Most sections are specific to scanning methods, except two sections.

The main configuration directive is \texttt{batconfig}. In this
section various global settings are defined, such as \texttt{multiprocessing}
which sets whether or not multiple CPUs (or cores) should be used during
scanning. The default configuration as shipped in the official BAT distribution
is to use multiple threads:

\begin{verbatim}
[batconfig]
multiprocessing = yes
\end{verbatim}

If set to \texttt{yes} the program will start an extra process per CPU that is
available for parts of the program that can be run in parallel. In most cases
it is completely safe to use multiprocessing.

It might be desirable to not use all processors on a machine, for example if
there are multiple scans of BAT running at the same time, or if other tasks
need to run on the machine. It is possible to set the maximum amount of
processors to use using the \texttt{processors} option:

\begin{verbatim}
processors = 2
\end{verbatim}

Another setting in this section is \texttt{outputlite}:

\begin{verbatim}
outputlite = yes
\end{verbatim}

It defaults to \texttt{yes}. If set to \texttt{yes} the output archive will omit
a full copy of the unpacked data, significantly decreasing the size of the
output archive, but making it harder to do a ``post mortem'' on the unpacked
data (a new analysis should be run to get it again). In previous versions of BAT
this setting defaulted to \texttt{no}.

There are two settings that determine where the code of the optional XML pretty
printer can be found:

\begin{verbatim}
module = bat.simpleprettyprint
output = prettyprintresxml
\end{verbatim}

These two settings should always be used together.

There is one setting to set the prefix for creating temporary files or
directories, namely \texttt{tempdir}. By default the directory for creating
temporary files and directories is \texttt{/tmp}. There might be situations
where the temporary directory might need to be changed, for example for
unpacking on a faster medium (ramdisk, SSD) than a normal harddisk. It can be
used as follows:

\begin{verbatim}
tempdir = /ssd/tmp
\end{verbatim}

To assist in debugging and finding errors in scans of BAT there are two
settings: \texttt{debug} and \texttt{debugphases}. The setting \texttt{debug}
can be used to enable and disable debugging. If set multiprocessing will be
disabled and information about which file is scanned and which method is run
will be printed on standard error. If specified without \texttt{debugphases}
this will apply to all scan phases. The \texttt{debugphases} parameter can be
used to limit this behaviour to just one or a few phases. The other phases will
behave normally. For example, this will enable debugging, but just for the
leaf scans and aggregate scans:

\begin{verbatim}
debug = yes
debugphases = leaf:aggregate
\end{verbatim}

The other global section is \texttt{viewer}. This section is specific for the
graphical frontend and is not used in any other parts of BAT and might be moved
to a separate configuration file in a future version of BAT.

\subsubsection{Enabling and disabling scans}

The standard configuration file enables most of the scans and methods
implemented in BAT by default. Scans can be enabled and disabled by setting the
option \texttt{enabled} to \texttt{yes} and \texttt{no} respectively.

Another way to not run a scan is to outcomment the entry in the configuration
file (by starting the line with the \texttt{\#} character), or by removing the
section from the configuration file.

\subsubsection{Blacklisting and whitelisting scans}

Files can be explicitely blacklisted for scanning by using the \texttt{noscan}
configuration setting. The value of this parameter is a list of tags, separated
by colons:

\begin{verbatim}
noscan      = text:xml:graphics:pdf:audio:video
\end{verbatim}

Similarly files can be whitelisted by using the \texttt{scanonly} setting. Only
files that are tagged with any of the values in this list (if not empty) will
be scanned. If there is an overlapping value in \texttt{scanonly} and
\texttt{noscan} then the file will not be scanned.

\subsubsection{Passing environment variables}

All scans have an optional parameter \texttt{envvars} defaulting to an empty
Python dictionary. In the configuration file a colon separated list of
name/value pairs can be specified. These will then become available in the
environment of the scan:

\begin{verbatim}
envvars     = BAT_REPORTDIR=/tmp/images:BAT_IMAGE_MAXFILESIZE=100000000
\end{verbatim}

It is up to the scans to parse the environment variables and to store them in
the environment.

\subsubsection{Storing results}

Postrun scans and aggregate scans that output data, for example graphics files
or reports, can specify which files should be added to the output file. There
are three settings that should be set together:

\begin{verbatim}
storetarget = images
storedir    = /tmp/images
storetype   = -piechart.png:-version.png
\end{verbatim}

The \texttt{storetarget} setting specifies the relative directory inside the
output TAR archive. The \texttt{storedir} setting tells where to get the files
that need to be stored can be found (this should be where the postrun scan
or aggregate scan stores its results). The \texttt{storetype} setting is a colon
separated list of extensions/partial file names that the files should end in
(typically the rest of the filename is a SHA256 value).

The additional setting \texttt{cleanup} can be used to instruct BAT that the
files generated by this postrun scan or aggregate scan should be removed after
copying them into the result archive:

\begin{verbatim}
cleanup     = yes
\end{verbatim}

The \texttt{cleanup} setting should be set to \texttt{yes} unless the results
do not change in between subsequent runs of BAT.

Currently (BAT 19) if \texttt{cleanup} is set the files are written directly to
output directories. The values of these directories are hardcoded (and match
values that the GUI expects) but these will be replaced by the value of
\texttt{storetarget} in a later release.

\subsubsection{Running setup code}

For some scans it is necessary to run some setup code to ensure that certain
conditions are met, for example that databases exist, or that locations are
readable/writeable. If this would be done during running of the scan (default
in BAT 13 and earlier) the checks would be run each time the scan would be run.
In reality the checks only need to run once.

There is a special hook for leaf scans to run setup code for the scan:

\begin{verbatim}
setup     = nameOfSetupMethod
\end{verbatim}

In the standard distribution of BAT the files \texttt{bat/identifier.py} and
\texttt{bat/licenseversion.py} contain very extensive examples of setup hooks.

\section{Analyser internals}

The analyser was written with extensibility in mind: new file systems or
variants of old ones tend to appear regularly (for example: there are at
least 5 or more versions of SquashFS with LZMA compression out there).

\subsection{Code organisation}

\texttt{bat-scan} is merely a frontend for the real scanner and only determine
the list of scans, the binary to scan and the output file.

The meaty bits of the analyser can be found in files in the \texttt{bat}
subdirectory.

\begin{itemize}
\item \texttt{bruteforcescan.py} contains the main logic of the program: it
launches scans based on what is inside the binary and the scans that are
enabled, collects results from scans and writes results to an output file.
\item \texttt{busybox.py} and \texttt{busyboxversion.py} contain code to
extract useful information from a BusyBox binary, such as the version number.
\item \texttt{checks.py} contains various leaf scans, like scanning for certain
marker strings, or the presence of license texts and URLs of
forges/collaborative software development sites.
\item \texttt{ext2.py} implements some functionality needed for unpacking ext2
file systems.
\item \texttt{extractor.py} provides convenience functions that are used
throughout the code.
\item \texttt{file2package.py} has code to match names of files to names of
packages from popular distributions in a database.
\item \texttt{findduplicates.py} is used to find duplicate files in the scanned
archive.
\item \texttt{findlibs.py} is for researching dynamically linked ELF files in
the archive.
\item \texttt{fsmagic.py} contains identifiers of various file systems and
compressed files, like magic headers and offsets for which might need to be
corrected.
\item \texttt{fwunpack.py} includes most of the functionality for unpacking
compressed files and file systems.
\item \texttt{generatehexdump.py} and \texttt{images.py} generate textual and
graphical representations of the input files.
\item \texttt{generatereports.py}, \texttt{generateimages.py},
\texttt{guireport.py} and \texttt{piecharts.py} generate textual and graphical
representations of results of the analysis.
\item \texttt{identifier.py} implements functionality to extract identifiers
(string constants, function names, method names, variable names, and so on)
from binary files and make them available for further analysis.
\item \texttt{jffs2.py} has code specific to handling JFFS2 file systems.
\item \texttt{kernelanalysis.py} includes code to extract information from
Linux kernel images and Linux kernel modules.
\item \texttt{licenseversion.py} is responsible for getting version and
license information for uniquely identified strings and function names (and in
the future variable names too), as well as optionally pruning the result set
to only include relevant versions. It also contains code to aggregate results
of Java class files from a JAR file and assign results to the JAR file.
\item \texttt{prerun.py} contains scans that are run in the pre-run phase for
correctly tagging files as early in the process as possible.
\item \texttt{simpleprettyprint.py} has the default XML prettyprinter.
\item \texttt{unpackrpm.py} has code specific to handling RPM archives.
\end{itemize}

\subsection{Pre-run methods}

Pre-run methods check and tag files, so the files can be ignored by later
methods and scans, reducing scanning time and preventing false positives. While
tagging is not exclusive to pre-run methods it is their main purpose.

\subsubsection{Writing a pre-run method}

Pre-run methods have a strict interface. Parameters are:

\begin{itemize}
\item \texttt{filename} is the absolute path of the file that needs to be tagged
\item \texttt{tempdir} is the (possibly) empty name of a directory where the
file is. This is currently unused and might be removed in the future.
\item \texttt{tags} is the set of tags that have already been defined for the
file.
\item \texttt{offsets} is the set of offsets that have been found for the file
\item \texttt{debug} is an environment variable that can be used to optionally
set the scan in debugging mode so it can print more information on standard
error. By default it is set to \texttt{False}.
\item \texttt{envvars} is an optionally empty set of environment variables that
can be used to pass extra information to the pre-run method.
\item \texttt{unpacktempdir} is the location of a directory for writing
temporary files. This value is optional and by default it is set to
\texttt{None}.
\end{itemize}

Return values are:

\begin{itemize}
\item a list containing tags
\end{itemize}

Example:

\begin{verbatim}
def prerunMethod(filename, tempdir=None, tags=[], offsets={},
                 debug=False, envvars=None, unpacktempdir=None):
        newtags = []
        newtags.append('helloworld')
        return newtags
\end{verbatim}

\subsection{Unpackers}

Unpackers are responsible for recursively unpacking binaries until they
can't be unpacked any further.

\subsubsection{Writing an unpacker}

The unpackers have a strict interface:

\begin{verbatim}
def unpackScan(filename, tempdir=None, blacklist=[], offsets={},
               debug=False, envvars=None):
        ## code goes here
\end{verbatim}

The last four parameters are optional, but in practice they are always passed
by the top level script.

\begin{itemize}
\item \texttt{tempdir} is the directory into which files and directories for
unpacking should be created. If it is \texttt{None} a new temporary directory
should be created.
\item \texttt{blacklist} is a list of byte ranges that should not be scanned.
If the current scan needs to blacklist a byte range it should add it to this
list after finishing a scan.
\item \texttt{offsets} is a dictionary containing a mapping from an identifier
to a list of offsets in the file where these identifiers can be found. This
list is filled by the scan \texttt{genericMarker} which always runs before
anything else.
\item \texttt{debug} is an environment variable that can be used to optionally
set the scan in debugging mode so it can print more information on standard
error. By default it is set to \texttt{False}.
\item \texttt{envvars} is an optionally empty set of environment variables that
can be used to pass extra information to the pre-run method.
\end{itemize}

Return values are:

\begin{itemize}
\item the name of a directory, containing files that were unpacked.
\item the blacklist, possibly appended with new values
\item a list of tags, in case any tags were added, or an empty list
\end{itemize}

Most scans have been split in two parts: one part is for searching the
identifiers, correctly setting up temporary directories and collecting results.
The other part is doing the actual unpacking of the data and verification.

The idea behind this split is that sometimes functionality is shared between
two scans. For example, \texttt{unpackCpio} is used by both
\texttt{searchUnpackCpio} and \texttt{unpackRPM}.

\subsubsection{Adding an identifier}

Identifiers for new file systems and compressed files are, if available, added
to \texttt{fsmagic.py} in the directory \texttt{bat}. These identifiers will be
available in the \texttt{offsets} parameter that is passed to a scan, if any
were found.

Good sources to find identifiers are \texttt{/usr/share/magic}, documentation
for file systems or compressed files, or the output of \texttt{hexdump -C}.

\subsubsection{Blacklisting and priorities}

In BAT blacklists are used to prevent some scans from running on a particular
byte range, because other scans have already covered these bytes, or will cover
them.

The most obvious example is the \texttt{ext2} file system: in a normal setup
(no encryption) it is trivial to see the content of all the individual files
if an \texttt{ext2} file system image is opened. This is because this file
system is mostly a concatenation of the data, with some meta data associated
with the files in the file system.

If another compressed file is in the \texttt{ext2} file system it could be that
it will be picked up by BAT twice: once it will be detected inside the
\texttt{ext2} file system and once after the file system has been unpacked by
the \texttt{ext2} file system unpacker.

Other examples are:

\begin{itemize}
\item cpio (files are concatenated with a header and a trailer)
\item TAR (files are concatenated with some meta data)
\item RPM (files are in a compressed archive with some meta data)
\item ar and DEB
\item some flavours of cramfs
\item ubifs
\end{itemize}

To avoid duplicate scanning and false positives it is therefore necessary to
prevent other scans from running on the byte range already covered by one of
these files.

In BAT this is achieved by using blacklists. All unpackers have a parameter
called \texttt{blacklist} which is consulted every time a file is unpacked. If
a file system offset is in a blacklist the scan could use the next offset, or
skip scanning the entire file, depending on the scan.

The blacklist is set for every file individually and is initially empty. If a
scan is successful it adds a byte range to the blacklist. Subsequent scans
will skip the byte range added by the scan.

The scans are run in a particular order to make the best use of blacklists. The
order of scans is determined by the \texttt{priority} parameter in the
configuration file. The file systems and concatenated files mentioned above
have a higher priority and are scanned earlier than other scans that could also
give a match. It is not a fool proof system, but it seems to work well enough.

\subsection{Leaf scans}

After everything has been unpacked each file, including the files from which
other files were carved, will be scanned by the leaf scans.

\subsubsection{Writing a leaf scan}

The leaf scans have a simple interface. There are six parameters passed to the
scan, namely the absolute path of the file, the tags of the file, an optional
blacklist with byte ranges that should not be scanned, an optional list of
environment variables and an optional name of a directory for writing temporary
results. For example:

\begin{verbatim}
def leafScan(path, tags, blacklist=[], debug=False,
                envvars=None, unpacktempdir=None):
        ## code goes here
\end{verbatim}

There are no restrictions on the return values of the leaf scan, except if
nothing could be found (in which case \texttt{None} is usd as return value).
The result value is a tuple with a list of tags as well as one of the following:

\begin{itemize}
\item \texttt{None} if nothing can be found
\item simple values (booleans, strings)
\item custom data structure. Code that processes this data should know about
its structure.
\end{itemize}

There is no restriction on the code that is run as part of the leaf scan and
basically anything can be done. In BAT there are for example checks that invoke
other external programs to discover dynamically linked libraries using
\texttt{readelf}, find the license of a kernel module using \texttt{modinfo}
or simple checks for the presence of strings in the binary that indicate the
use of certain software.

The simplest scans are the ones that search for hardcoded strings. These strings
are frequently found just in the package for which the check is written for. For
example, the following strings can often be found in copies of the
\texttt{iptables} binary and the related \texttt{libiptc} library:

\begin{verbatim}
markerStrings =
     [ 'iptables who? (do you need to insmod?)'
     , 'Will be implemented real soon.  I promise ;)'
     , 'can\'t initialize iptables table `%s\': \%s'
     ]
\end{verbatim}

Although searching for hardcoded strings is very fast, this method has some
drawbacks:

\begin{itemize}
\item a binary sometimes does not have these exact strings embedded
\item this method will only find the strings that are hardcoded and not any
other significant strings
\item if another package includes the string, it will be a false positive
\end{itemize}

The quick checks should therefore only be used as an indication that further
inspection of the binary is needed. A much better method is the ranking method
that is also available in BAT, but which requires a special setup with a
database.

\subsubsection{Pretty printing for leaf scans}

Pretty printing for unpackers is standardized but for leaf scans there is
more flexibility. This is needed because in some cases the result as returned
by the leaf scan needs post processing due to use of custom data structures.

A pretty printer can be defined in the configuration by setting
\texttt{ppoutput}. The pretty printer can be in the same module as the scanning
method defined in the same section, but does not need to be. If it resides in
another module it can be set using \texttt{ppmodule}. The pretty printer has two
parameters: a Python datastructure as returned by the scanner (this differs
per scan) and a XML root element, needed to create new XML nodes. The method is
expected to return a XML node in case of success, or \texttt{None} in case of
failure.

If no pretty printer is defined the value as returned by the scan will be used
as the content of result tag.

\subsection{Aggregators}

Aggregators take all information from the entire scan process and possibly
modify results.

\subsubsection{Writing an aggregator}

Aggregators have a strict interface:

\begin{verbatim}
def aggregateexample(unpackreports, scantempdir, topleveldir, debug=False,
                     envvars=None, unpacktempdir=None)
\end{verbatim}

\begin{itemize}
\item \texttt{unpackreports} are the reports of the unpackers for all files
\item \texttt{scantempdir} is the location of the top level data directory of
the scan
\item \texttt{topleveldir} is the location of the top level directory of the
scan
\item \texttt{debug} is an environment variable that can be used to optionally
set the scan in debugging mode so it can print more information on standard
error. By default it is set to \texttt{False}.
\item \texttt{envvars} is an optional dictionary of environment variables
\item \texttt{unpacktempdir} is the location of a directory for writing
temporary files. This value is optional and by default it is set to
\texttt{None}.
\end{itemize}

The aggregators should read any results of the leaf scans from the pickles
on disk.

If there is any result it should be returned as a dictionary with one key. It
will be assigned to the results of the top level element. Examples are: the
names of files which are duplicates in an archive or firmware.

\subsection{Post-run methods}

Post-run methods don't change the result of the whole scanning process, but
only use the data from the process. For example prettyprinting a fancy report
(more advanced than the standard XML report) would be a typical post-run
method.

\subsubsection{Writing a post-run method}

Post-run methods have a strict interface:

\begin{verbatim}
def postrunHelloWorld(filename, unpackreport, scantempdir, topleveldir,
                      debug=False, envvars={}):
        print "Hello World"
\end{verbatim}

\begin{itemize}
\item \texttt{filename} is the absolute path of the scanned file, after
unpacking.
\item \texttt{unpackreport} is the report of unpacking the file
\item \texttt{scantempdir} is the directory that contains the unpacked data
\item \texttt{topleveldir} is the top level directory containing the data
directory and the directory with the per file result pickles.
\item \texttt{debug} is an environment variable that can be used to optionally
set the scan in debugging mode so it can print more information on standard
error. By default it is set to \texttt{False}.
\item \texttt{envvars} is an optional dictionary of environment variables
\end{itemize}

The post-run methods should read any results of the leaf scans from the pickles
stored on disk. Since the post-run methods don't change the result in any way,
but just have side effects there is no need to return anything. Any return value
will be ignored.

\section{Building binary packages of the Binary Analysis Tool}

If you want to install BAT through the package manager of your distribution you
might first need to generate packages for your distribution if none exist. For
BAT there is currently support to build packages for RPM-based systems and for
DEB-based systems.

\subsection{Building packages for RPM based systems from releases}

Building RPMs from released versions of BAT is trivial: download the SRPM
files for \texttt{bat}, \texttt{bat-extratools} and
\texttt{bat-extratools-java} from the BAT website and rebuild them with
\texttt{rpmbuild --rebuild}.

\subsection{Building packages for RPM based systems from Subversion}

\subsubsection{Building \texttt{bat}}

Building the \texttt{bat} package is fairly straightforward.

\begin{enumerate}
\item Make a fresh export of BAT from Subversion
\item run the command: \texttt{python setup.py bdist\_rpm}
\end{enumerate}

This will create an RPM file and an SRPM file. If you need to install BAT on
other versions of Fedora or on other RPM based distributions you can simply
rebuild the SRPM using \texttt{rpmbuild --rebuild}.

\subsubsection{Building \texttt{bat-extratools} and
\texttt{bat-extratools-java}}

Building packages for \texttt{bat-extratools} and \texttt{bat-extratools-java}
is unfortunately a bit more elaborate.

\begin{enumerate}
\item make a fresh export of the Subversion repository
\item change the names of \texttt{bat-extratools} and the
\texttt{bat-extratools-java} directories to contain the version name of the
release (for example \texttt{bat-extratools-14.0}). Make a \texttt{tar.gz}
archive of the directory:
\texttt{tar zcf bat-extratools-14.0.tar.gz bat-extratools-14.0}
\item run \texttt{rpmbuild} to create binary packages:
\texttt{rpmbuild -ta bat-extratools-14.0.tar.gz}
\end{enumerate}

\subsection{Building packages for DEB based systems from releases}

Currently no rebuildable packages for DEB based systems are made for releases.

\subsection{Building packages for DEB based systems from Subversion}

\subsubsection{Building \texttt{bat}}

The Debian scripts were written according to the documentation for
\texttt{debhelper} found at \url{https://wiki.ubuntu.com/PackagingGuide/Python}.

Package building and testing is done on Ubuntu 14.04 LTS. Older versions of
Ubuntu are no longer supported and its use is discouraged. This is because
versions of Ubuntu older than 14.04 use a broken version of the PyDot package.

To build a \texttt{.deb} package do an export of the Subversion repository
first.  Change to the directory \texttt{src} and type:
\texttt{debuild -uc -us} to build the package. This assumes that you will have
the necessary packages installed to build the package (like \texttt{devscripts}
and \texttt{debhelper}).

The build process might complain about not being able to find the original
sources. In our experience it is safe to ignore this. The command will build a
\texttt{.deb} package which can be installed with \texttt{dpkg -i}.

\subsubsection{Building \texttt{bat-extratools} and
\texttt{bat-extratools-java}}

To build a .deb package do an export of the Subversion repository first. Change
to the correct directories (\texttt{bat-extratools} and 
\texttt{bat-extratools-java} and type: \texttt{debuild -uc -us} to
build the packages.

\section{Binary Analysis Tool knowledgebase}

BAT comes with a mechanism to use a database backend. The default version of
BAT only unpacks file systems and compressed files and runs a few simple checks
on the leaf nodes of the unpacking process.

In the paper ``Finding Software License Violations Through Binary Code Clone
Detection'' by Hemel et. al. (ACM 978-1-4503-0574-7/11/05), presented at
the Mining Software Repositories 2011 conference, a method to use a database
with strings extracted from source code was described. This functionality is
available in the ranking module in the file \texttt{licenseversion.py}. This
code is enabled by default, but if no database is present it will not do
anything.

To give good results the database that is used needs to be populated with as
many packages as possible, from a cross cut of all of open source software, to
prevent bias towards certain packages: if you only would have BusyBox in your
database, everything would look like BusyBox.

If you don't want to spend much time on downloading and processing
packages, please contact Tjaldur Software Governance Solutions for
purchasing a copy of a fully prepared database at \url{info@tjaldur.nl}.

\subsection{Generating the package list}

The code and license extractor wants a description file of which packages to
process. This file is hardcoded to \texttt{LIST} relative to the directory that
contains all source archives. The reason there is a specific file is that some
packages do not follow a consistent naming scheme. By using this extra file we
can cleanup names and make sure that source code archives are recognized
correctly.

The file contains four values per line:

\begin{itemize}
\item name
\item version
\item archivename
\item origin (defaults to ``unknown'' if not specified)
\end{itemize}

separated by whitespace (spaces or tabs). An example would look like this:

\begin{verbatim}
amarok	2.3.2	amarok-2.3.2.tar.bz2	kde
\end{verbatim}

This line says that the package is \texttt{amarok}, the version number is
\texttt{2.3.2}, the filename is \texttt{amarok-2.3.2.tar.bz2} and the file
was downloaded from the KDE project.

There is a helper script (\texttt{generatelist.py}) to help generate the file.
It can be invoked as follows:

\begin{verbatim}
python generatelist.py -f /path/to/directory/with/sources -o origin
\end{verbatim}

The output is printed on standard output, so you want to redirect it to a file
called \texttt{LIST} (as expected by the string extraction script) and
optionally sorting it first:

\begin{verbatim}
python generatelist.py -f /path/to/directory/with/sources
    -o origin | sort > /path/to/directory/with/sources/LIST
\end{verbatim}

\texttt{generatelist.py} tries to determine the name of the package by
splitting the file name on the right on a \texttt{-} (dash)
character. This is not always done correctly because a package uses multiple
dashes, or because it does not contain a dash. In the latter case an error
will be printed on standard error, informing you that a file could not be
added to the list of packages and it should be added manually.

It is advised to manually inspect the file after generating it to ensure the
correctness of the package names. Packages can have been renamed for a number
of reasons:

\begin{itemize}
\item upstream projects decided to use a new name for archives (AbiWord
archives for example were renamed from \texttt{abi-\$VERSION.tar.gz} (used for
early versions) to \texttt{abiword-\$VERSION.tar.gz}).
\item a distribution has renamed packages to avoid clashes during installation
and allow different versions to be installed next to eachother.
\item a distribution has renamed a package. For example, Debian renamed
\texttt{httpd} to \texttt{apache2}.
\end{itemize}

In these cases you need to change the names of the packages, otherwise
different versions of the same package will be recorded in the database as
different packages, which will confuse the rating algorithm and cause it to
give suboptimal results.

Other helper scripts are \texttt{dumplist.py} which recreates a package list
file from a database, and \texttt{rewritelist.py} which takes two package list
files and outputs a new file with package names and versions rewritten for
filenames that occur in both files. These two scripts are useful if a database
needs to be regenerated, possibly with new packages.

\subsection{Creating the database}

The program to extract strings from sourcecode is \texttt{createdb.py}. It is
not part of the standard installation of BAT, but needs to be retrieved
separately from version control together with \texttt{generatelist.py}. This
will be changed at some point in the future.

It parses the file generated by \texttt{generatelist.py}, unpacks the files
(gzip compressed TAR, bzip2 compressed TAR, LZMA compressed TAR< XZ compressed
TAR and ZIP are currently supported) and scans each individual source code file
(written in C, C++, assembler, QML, C\#, Java, Scala, JSP, Groovy, PHP, Python
and ActionScript) for string constants, methods, functions, variables and, if
enabled, licenses using Ninka and FOSSology and copyright information using
FOSSology.

For the Linux kernel additional information is extracted about kernel functions
and variables, module information (author, license, parameters, and so on),
and kernel symbol information.

\texttt{createdb.py} can be invoked as follows:

\begin{verbatim}
python createdb.py -f
    /path/to/directory/with/files -c /path/to/configurationfile
\end{verbatim}

The configuration file is a simple configuration file in Windows INI format. An
example of a configuration file is as follows:

\begin{verbatim}
[extractconfig]
configtype = global
database = /home/bat/db/master.sqlite3
scanlicense = yes
licensedb = /home/bat/db/licenses.sqlite3
ninkacommentsdb = /home/bat/db/ninkacomments.sqlite3
scancopyright = yes
cleanup = yes
wipe = no
unpackdir = /ramdisk
\end{verbatim}

The global section is called \texttt{extractconfig}. The field
\texttt{configtype} has to be set to \texttt{global}. The field
\texttt{database} is used to set the path to the main database. This parameter
is mandatory: if it is not set the script will exit. The parameters
\texttt{scanlicense} and \texttt{scancopyright} can be used to enable or
disable license and copyright scanning (default: disabled). \texttt{licensedb}
is used to set the path to the copyright and licensing database. The parameter
\texttt{ninkacommentsdb} can be used for setting a caching database for mapping
comments to licenses, as used by Ninka. This setting is mandatory if license
scanning is enabled.

If \texttt{cleanup} is set to \texttt{yes} (default) the temporary directory
with unpacked sources will be removed. If \texttt{wipe} (default: \texttt{no})
is set to \texttt{yes} all tables and indexes will first be dropped. The
parameter \texttt{unpackdir} can be used to set a location where archives are
unpacked, for example a ramdisk or SSD.

Apart from the global section there are also package specific sections:

\begin{verbatim}
[bash]
configtype = package
extensions = .def:C
\end{verbatim}

The section name (in the example \texttt{bash}) is the name of the package and
is used by \texttt{createdb.py} to match with a package name. The field
\texttt{configtype} should be set to \texttt{package}. The only field is
\texttt{extensions} which defines pairs of extensions and languages for files
with package specific extensions that are interesting to scan. For example
\texttt{bash} has quite a few strings that end up in binaries defined in its
source tree that end on \texttt{.def}. These files are only interesting in the
context of \texttt{bash}. An extension/language pair has a semicolon as a
separator. Multiple pairs are separated by whitespace.

\subsection{License extraction and copyright information extraction}

The configuration for \texttt{createdb.py} has a few options. The most important
ones to consider are whether or not to also extract licenses and copyrights from
the source code files.  License extraction is done using the Ninka license
scanner and the Nomos license scanner from FOSSology. Copyright scanning is done
using the copyright scanner from FOSSology. These options are disabled
by default for a few reasons:

\begin{itemize}
\item extracting licenses and copyrights costs significantly more time
\item there are no packages for Fedora and Debian/Ubuntu for Ninka
\end{itemize}

If you want to enable license extraction, you will have to install Ninka first
and change one hardcoded path that points to the main Ninka script in
\texttt{createdb.py}. You will also have to install FOSSology
(for which packages are available for most distributions).

\subsection{Configuring identifier extraction}

\begin{verbatim}
[identifier]
type        = leaf
module      = bat.identifier
method      = searchGeneric
envvars     = BAT_NAMECACHE_C=/home/bat/db/functioncache_c:
              DEX_TMPDIR=/ramdisk:BAT_STRING_CUTOFF=5
noscan      = text:xml:graphics:pdf:compressed:
              resource:audio:video:mp4:vimswap:timezone:ico
description = Classify packages using advanced ranking mechanism
enabled     = yes
setup       = extractidentifiersetup
priority    = 1
\end{verbatim}

The three parameters are:

\begin{itemize}
\item \texttt{BAT\_NAMECACHE\_C} - in case the binary is a Linux kernel image
or Linux kernel module the identifier extraction scan needs to be able to look
up kernel function names to filter these out and assign to the function
results. For this a caching database with function names is needed.
\item \texttt{BAT\_STRING\_CUTOFF} - this value is the mimimal length of the
string that is matched (default value is 5). If extracted strings are shorter
than this value they will be ignored. It is important to keep this parameter in
sync with the minimum length of strings in the database extract script.
\item \texttt{DEX\_TMPDIR} - set the location of a temporary location for
unpacking Android DEX files. This can be for example set to the location of a
ramdisk.
\end{itemize}

\subsection{Configuring the ranking method}

The ranking method can be found in \texttt{bat/licenseversion.py}. The ranking
method looks up strings in the database, optionally aggregates results for
Java class files at the JAR level, determines versions and licenses while also
removing unlikely versions from the result set.

The ranking method uses a few databases. For the first part (determining which
package a string belongs to) it uses specialised caching databases for string
constants, function names, variable names and so on. These caching databases
contain a subset of information to vastly speed up scanning. There is no script
in the standard distribution of BAT to create these caching databases.

The second part (determining versions and licenses) the full database is used.

The location of the databases can be set in the configuration file in
the \texttt{envvars} option:

\begin{verbatim}
[versionlicensecopyright]
type        = aggregate
module      = bat.licenseversion
method      = determinelicense_version_copyright
noscan      = text:xml:graphics:pdf:audio:video:mp4
envvars     = BAT_DB=/home/bat/db/master.sqlite3:
              BAT_LICENSE_DB=/home/bat/db/licenses.sqlite3:
              BAT_CLONE_DB=/home/bat/db/clonedb.sqlite3:
              BAT_STRINGSCACHE_C=/home/bat/db/stringscache_c:
              BAT_STRINGSCACHE_JAVA=/home/bat/db/stringscache_java:
              BAT_NAMECACHE_C=/home/bat/db/functioncache_c:
              BAT_NAMECACHE_JAVA=/home/bat/db/functioncache_java:
              BAT_STRING_CUTOFF=5:AGGREGATE_CLEAN=1:USE_SOURCE_ORDER=1:
              BAT_RANKING_LICENSE=1:BAT_RANKING_VERSION=1:
              BAT_KEEP_VERSIONS=10:BAT_KEEP_MAXIMUM_PERCENTAGE=50:
              BAT_MINIMUM_UNIQUE=10
enabled     = yes
priority    = 3
\end{verbatim}

The entire database with all other information except license information is
\texttt{BAT\_DB}. This option is mandatory. If it is not supplied scanning with
the determining versions and licenses will fail.

In the database the strings, averages, function names, variable names, etcetera
are split per language family (C, Java, C\#, and so on). The reason for this is
that strings/function names that are very significant in one programming
language family could be very generic in another programming language family
and vice versa. During scanning a guess will be made to see which language the
program was written in and the proper caching database will be queried.

Since there are relatively few binaries (at least on Linux) that combine
code from both languages the caching databases are split. This makes the
caching databases a lot smaller so they can easier fit into memory. There are
of course programs with language embeddeding and better support for these will
be added in the future.

The names of the caching databases start with \texttt{BAT\_STRINGSCACHE} and
\texttt{BAT\_NAMECACHE} and are postfixed with an underscore and the name of
the programming language family.  The strings cache database for Java for
example is configured using the environment variable
\texttt{BAT\_STRINGSCACHE\_JAVA}.

An optional database to deal with copied and renamed packages can be set with
\texttt{BAT\_CLONE\_DB}. If set and populated the ranking scan will use
information from this database to rewrite package names. This is useful if a
package was renamed for a reason and different packages should be treated as if
they were a single package. Examples are Ethereal that had to be renamed to
Wireshark, or KOffice that was forked into Calligra, after which development
on KOffice effectively stopped and everyone moved to Calligra.

The license database can be set with \texttt{BAT\_LICENSE\_DB}. If it is not
supplied licensing information will not be used during the scan. If
\texttt{BAT\_RANKING\_LICENSE} is not set to 1 no license information will be
extracted. If \texttt{BAT\_RANKING\_VERSION} is not set to 1 no version
information will be extracted. If \texttt{BAT\_RANKING\_LICENSE} is set to 1 it
automatically sets \texttt{BAT\_RANKING\_VERSION} to 1 as well.

The parameter \texttt{USE\_SOURCE\_ORDER} can be used to tell the matching
algorithm to assume that identifiers in the binary code are similar as in the
source code and that the compiler has not reordered these. As compilers often
keep the order this assigns more strings to packages. As soon as compilers
start reordering identifiers this method will not work. The default setting
is to not use the order of identifiers.

The parameter \texttt{BAT\_STRING\_CUTOFF} indiciates the mimimal length of the
string that is matched (default value is 5). If extracted strings are shorter
than this value they will be ignored. It is important to keep this parameter in
sync with the minimum length of strings in the database extract script.

Results of Java class files are aggregated per JAR where the class files were
found in. If the parameter \texttt{AGGREGATE\_CLEAN} is set to 1 the class
files will be removed from the result set after aggregating the results. By
default class files will not be removed.

The parameters \texttt{BAT\_KEEP\_VERSIONS}, \texttt{BAT\_MINIMUM\_UNIQUE} and
\texttt{BAT\_KEEP\_MAXIMUM\_PERCENTAGE} are used to tell the pruning methods
how many versions to keep, how many unique strings minimally should be found,
and so on.

\subsubsection{Interpreting the results}

There are two ways to interpret the results. The recommended way is to load the
result file into the graphical user interface. The other way is to have BAT
pretty print the result in XML and further process the XML file.

The results of the scan can be found in the element \texttt{<ranking>}. This
element contains:

\begin{itemize}
\item number of lines that were extracted from the binary
\item number of lines that could be matched exactly with an entry in the
database
\item result per package which are a possible match
\end{itemize}

Per package the following is reported:

\begin{itemize}
\item name of the package
\item all unique matches (strings that can only be found in this package)
\item relative ranking
\item percentage of the total score
\end{itemize}

For example, take the results of a run on a BusyBox binary:

\begin{verbatim}
<ranking>
  <matchedlines>1314</matchedlines>
  <extractedlines>3147</extractedlines>
  <package>
    <name>busybox</name>
    <uniquematches>
        <unique>%d heads, %d sectors/track, %d cylinders</unique>
        ...
    </uniquematches>
    <rank>1</rank>
    <percentage>98.3386895181</percentage>
  </package>
  ...
</ranking>
\end{verbatim}

About 98\% of the total score was for BusyBox, so it is a clear match. In
programs were two or more packages are embedded percentages will be distributed
in a different, more uniform, way.

\subsection{Database design}

The main database currently has 14 tables, 9 of which are Linux kernel specific:

\begin{itemize}
\item \texttt{processed}
\item \texttt{processed\_file}
\item \texttt{extracted\_file}
\item \texttt{extracted\_function}
\item \texttt{extracted\_name}
\item \texttt{kernel\_configuration}
\item \texttt{kernelmodule\_alias}
\item \texttt{kernelmodule\_author}
\item \texttt{kernelmodule\_description}
\item \texttt{kernelmodule\_firmware}
\item \texttt{kernelmodule\_license}
\item \texttt{kernelmodule\_parameter}
\item \texttt{kernelmodule\_parameter\_description}
\item \texttt{kernelmodule\_version}
\end{itemize}

The licenses database has 2 tables:

\begin{itemize}
\item \texttt{extracted\_copyright}
\item \texttt{licenses}
\end{itemize}

During creation an additional table \texttt{ninkacomments} is used, but this
is only used to cache licensing information determined by the Ninka license
scanner. It is not used otherwise.

\subsubsection{\texttt{processed} table}

This table is to keep track of which versions of which packages were scanned.
Its only purpose is to avoid scanning packages multiple times. It is not
actively used in the ranking code.

It has the following fields:

\begin{itemize}
\item \texttt{package}: name of the package
\item \texttt{version}: version of the package
\item \texttt{filename}: name of the archive
\item \texttt{origin}: site where the archive was downloaded (optional)
\item \texttt{sha256}: SHA256 checksum of the archive
\end{itemize}

\subsubsection{\texttt{processed\_file} table}
This table contains information about of individual source code files that were
scanned.

It has the following fields:

\begin{itemize}
\item \texttt{package}: name of the package the file is from (same as in
\texttt{processed})
\item \texttt{version}: version of the package the file is from (same as in
\texttt{processed})
\item \texttt{filename}: relative path inside the source code archive
\item \texttt{sha256}: SHA256 checksum of the file
\end{itemize}

\subsubsection{\texttt{extracted\_file} table}
This table stores the individual strings that were extracted from files and
that could possibly end up in binaries.

It has the following fields:

\begin{itemize}
\item \texttt{programstring}: string constant that was extracted
\item \texttt{sha256}: SHA256 checksum of file the string constant was
extracted from
\item \texttt{language}: language the source code file was written in (mapped
to a language family, such as C or Java)
\item \texttt{linenumber}: line number where the string constant can be found
in the source code file (if determined using using \texttt{xgettext}) or $0$
(if determined using a regular expression).
\end{itemize}

\subsubsection{\texttt{extracted\_function} table}

In this table information about C functions and Java methods is stored.

\begin{itemize}
\item \texttt{sha256}: SHA256 checksum of the file
\item \texttt{functionname}: function name or method name that was extracted
\item \texttt{language}: language the source code file was written in (mapped
to a language family, such as C or Java)
\item \texttt{linenumber}: line number where the function/method can be found
in the source code file (if determined using using \texttt{xgettext}) or $0$
(if determined using a regular expression).
\end{itemize}

\subsubsection{\texttt{extracted\_name} table}
This table stores information of various names extracted from source code.
Included are variable names (C), field names (Java) and class names (Java) and
Linux kernel variable names.

It has the following fields:

\begin{itemize}
\item \texttt{sha256}: SHA256 checksum of the file
\item \texttt{name}: name of variable, field or class name that was extracted
\item \texttt{type}: type (field, variable, class name, etcetera)
\item \texttt{language}: language the source code file was written in (mapped
to a language family, such as C or Java)
\item \texttt{linenumber}: line number where the function/method can be found
in the source code file (if determined using using \texttt{xgettext}) or $0$
(if determined using a regular expression).
\end{itemize}

\subsubsection{\texttt{extracted\_copyright} table}
This table stores copyright information that was extracted from files by
FOSSology.

It has the following fields:

\begin{itemize}
\item \texttt{sha256}: SHA256 checksum of the file
\item \texttt{copyright}: copyright information that was extracted
\item \texttt{type}: type of information that was extracted, currently
\texttt{url}, \texttt{email} or \texttt{statement}
\item \texttt{offset}: byte offset in the file where the copyright statement
can be found
\end{itemize}

\subsubsection{\texttt{kernel\_configuration} table}

The Makefiles in the Linux kernel configuration contain a lot of information
about which configuration includes which files. This information can be used to
reconstruct a possible kernel configuration that was used to create the Linux
binary image. The table has the following fields:

\begin{itemize}
\item \texttt{configstring}: configuration directive in Linux kernel
\item \texttt{filename}: filename/directory name to which the configuration
directive applies
\item \texttt{version}: Linux kernel version
\end{itemize}

\subsubsection{\texttt{kernelmodule\_alias} table}

This table is used to store information about Linux kernel module aliases. This
information is declared in the Linux kernel source code using the
\texttt{MODULE\_ALIAS} macro. The table has the following fields:

\begin{itemize}
\item \texttt{sha256}: SHA256 checksum of the file
\item \texttt{modulename}: name of the source code file
\item \texttt{alias}: contents of the \texttt{MODULE\_ALIAS} macro
\end{itemize}

\subsubsection{\texttt{kernelmodule\_author} table}

This table is used to store information about Linux kernel module author(s).
This information is declared in the Linux kernel source code using the
\texttt{MODULE\_AUTHOR} macro. The table has the following fields:

\begin{itemize}
\item \texttt{sha256}: SHA256 checksum of the file
\item \texttt{modulename}: name of the source code file
\item \texttt{author}: contents of the \texttt{MODULE\_AUTHOR} macro
\end{itemize}

\subsubsection{\texttt{kernelmodule\_description} table}

This table is used to store information about Linux kernel module descriptions.
This information is declared in the Linux kernel source code using the
\texttt{MODULE\_DESCRIPTION} macro. The table has the following fields:

\begin{itemize}
\item \texttt{sha256}: SHA256 checksum of the file
\item \texttt{modulename}: name of the source code file
\item \texttt{description}:
\end{itemize}

\subsubsection{\texttt{kernelmodule\_firmware} table}

This table is used to store information about Linux kernel module firmware.
This information is declared in the Linux kernel source code using the
\texttt{MODULE\_FIRMWARE} macro. The table has the following fields:

\begin{itemize}
\item \texttt{sha256}: SHA256 checksum of the file
\item \texttt{modulename}: name of the source code file
\item \texttt{firmware}: contents of the \texttt{MODULE\_FIRMWARE} macro
\end{itemize}

\subsubsection{\texttt{kernelmodule\_license} table}

This table is used to store information about Linux kernel module licenses.
This information is declared in the Linux kernel source code using the
\texttt{MODULE\_LICENSE} macro. The table has the following fields:

\begin{itemize}
\item \texttt{sha256}: SHA256 checksum of the file
\item \texttt{modulename}: name of the source code file
\item \texttt{license}: contents of the \texttt{MODULE\_LICENSE} macro
\end{itemize}

\subsubsection{\texttt{kernelmodule\_parameter} table}

This table is used to store information about Linux kernel module parameters.
This information is declared in the Linux kernel source code using the
\texttt{MODULE\_PARM} and \texttt{module\_param} macros, as well as variations
of the \texttt{module\_param} macro. These different notations were used for
different versions of the Linux kernel and both formats have been used in the
kernel at the same time. The table has the following fields:

\begin{itemize}
\item \texttt{sha256}: SHA256 checksum of the file
\item \texttt{modulename}: name of the source code file
\item \texttt{paramname}: name of the parameter
\item \texttt{paramtype}: type of the parameter, as specified in the source
code (various formats have been used)
\end{itemize}

\subsubsection{\texttt{kernelmodule\_parameter\_description} table}

This table is used to store information about Linux kernel module parameters
descriptions.  This information is declared in the Linux kernel source code
using the \texttt{MODULE\_PARM\_DESC} macro. The table has the following fields:

\begin{itemize}
\item \texttt{sha256}: SHA256 checksum of the file
\item \texttt{modulename}: name of the source code file
\item \texttt{paramname}: name of the parameter
\item \texttt{description}: descriptio of the parameter
\end{itemize}

\subsubsection{\texttt{kernelmodule\_version} table}

This table is used to store information about Linux kernel module versions.
This information is declared in the Linux kernel source code using the
\texttt{MODULE\_VERSION} macro. The table has the following fields:

\begin{itemize}
\item \texttt{sha256}: SHA256 checksum of the file
\item \texttt{modulename}: name of the source code file
\item \texttt{version}: contents of the \texttt{MODULE\_VERSION} macro
\end{itemize}

\subsubsection{\texttt{licenses} table}

This table stores the licenses that were extracted from files using a source
code scanner, like Ninka or FOSSology. If a file has more than one licenses
there will be multiple rows for a file. It has these fields:

\begin{itemize}
\item \texttt{sha256}: SHA256 checksum of the file
\item \texttt{license}: license as found by the scanner
\item \texttt{scanner}: scanner name. Currently only Ninka and FOSSology are
used in BAT, but is not limited to that: the scanner could also be a person
doing a manual review.
\item \texttt{version}: version of scanner. This is useful if there is for
example a bug in a scanner, or to compare results from various versions.
\end{itemize}

\section{BusyBox script internals}

The BusyBox processing scripts look simple, but behind the internals are a bit
hairy. Especially extracting the correct configuration is not trivial.

\subsection{Detecting BusyBox}

Detecting if a binary is indeed BusyBox is trivial, since in a BusyBox binary
there are almost always clear indication strings if BusyBox is used (unless
they it was specifically altered to hide the use of BusyBox).

A significant set of strings to look for is:

\begin{verbatim}
        BusyBox is a multi-call binary that combines many common Unix
        utilities into a single executable.  Most people will create a
        link to busybox for each function they wish to use and BusyBox
        will act like whatever it was invoked as!
\end{verbatim}

Another clear indicator is a BusyBox version string, for example:

\begin{verbatim}
BusyBox v1.15.2 (2009-12-03 00:14:42 CET)
\end{verbatim}

As an exception a BusyBox binary configured to include just a single applet
will not contain contain the marker strings, or the BusyBox version string. In
such a case a different detection mechanism will have to be used, for example
the ranking code as used by \texttt{bat-scan}, although this will only be
necessary in a very small percentage of cases, since the vast majority of
BusyBox instances include more than one applet.

\subsection{BusyBox version strings}

The BusyBox version strings have remained fairly consistent over the years:

\begin{verbatim}
BusyBox v1.00-rc2 (2006.09.14-03:08+0000) multi-call binary
BusyBox v1.1.3 (2009.09.11-12:49+0000) multi-call binary
BusyBox v1.15.2 (2009-12-03 00:14:42 CET)
\end{verbatim}

The time stamps in the version string are irrelevant, since they are generated
during build time and are not hardcoded in the source code.

Extracting version information from the BusyBox binary is not difficult.
Using regular expression it is possible to look for \texttt{BusyBox v} which
indicates the start of a BusyBox version string. The version number can be
found immediately following this substring until \texttt{ (} (including
leading space) is found.

Apart from reporting, the BusyBox version number is also used for other
things, such as determining the right configuration format and accessing a
knowledgebase of known applet names extracted from the standard BusyBox
releases from \url{busybox.net}.

\subsection{BusyBox configuration format}

During the compilation of BusyBox a configuration file is used to determine
which functionality will be included in the binary. The format of this
configuration file has changed a few times over the years. Early versions used
a simple header format file, with GNU C/C++ style defines. Later versions,
starting 1.00pre1, moved to Kbuild, the same configuration system as used by
for example the Linux kernel or OpenWrt. This format is still in use today
(BusyBox 1.20.0 being the latest version at the time of writing).

Each configuration directive determines whether or not a certain piece of
source code will be compiled and up in the BusyBox binary. This source code can
either be a full applet, or just a piece of functionality that merely extends
an existing applet.

\subsection{Extracting a configuration from a BusyBox binary}

Extracting the BusyBox configuration from a binary is not entirely trivial.
There are a few methods which can be used:

\begin{enumerate}
\item run \texttt{busybox} (on a device, or inside a sandbox) and see what
functionality is reported. This is probably the most accurate method, but also
the hardest, since it requires access to a device, or a sandbox that has been
properly set up, with all the right dependencies, and so on.

When running \texttt{busybox} without any arguments, or with the \texttt{--help}
parameter it will output a list of functions that are defined inside the
binary:

\begin{verbatim}
Currently defined functions:
        ar, cal, cpio, dpkg, dpkg-deb, gunzip, zcat
\end{verbatim}

These can be mapped to a configuration, using information extracted from
BusyBox source code about which applets map to which configuration option.
\item extract the configuration from the binary by searching for known applet
names in the firmware. The end result is the same as a previous step, but
possibly with less accuracy in some cases but it is the only feasible solution
if you only have a binary.
\end{enumerate}

The BusyBox binary has a string embedded for every applet that is included.
This is the string that is printed out if \texttt{--help} is given as a
parameter to an invocation of \texttt{busybox}.

Using information about the configuration extracted from BusyBox source code
these strings can be mapped to a configuration directive and a possible
configuration can be reconstructed.

Depending on how the binary was compiled this can be trivial, or quite hard.

\subsubsection{BusyBox linked with uClibc}

In binaries that link against uClibc (a particular C library) the name of the
main function of the applet is sometimes (but not always) included in the
\texttt{busybox} binary as follows (a good way is to run \texttt{strings} on
the binary and look at the output).

\begin{verbatim}
wget_main
\end{verbatim}

This string maps to the name of the main function for the \texttt{wget} applet
(\texttt{networking/wget.c}):

\begin{verbatim}
int wget_main(int argc, char **argv) MAIN_EXTERNALLY_VISIBLE;
\end{verbatim}

The BusyBox authors are pretty strict in their naming and usually have a
configuration directive in the a specific format
(\texttt{CONFIG-\$appletname}) in the Makefile, like:

\begin{verbatim}
lib-$(CONFIG_WGET)         += wget.o
\end{verbatim}

(example taken from \texttt{networking/Kbuild} in BusyBox 1.15.2). There are
cases where the format could be slightly different.

\subsubsection{BusyBox linked with glibc \& uClibc exceptions}

Sometimes the method described in the previous section does not work for
binaries that are linked with uClibc. It also does not work with binaries
compiled with glibc.

If the binary is unstripped and the binary still contains symbol information
it is possible to extract the right information using \texttt{readelf} (part
of GNU binutils) in a similar fashion as the earlier described method.

In case there is no information available it is still possible to search inside
the binary for the applet names. Because most instances of BusyBox that are
installed on devices have not been modified the list of applets in the stock
version of BusyBox serves as an excellent starting point.

The list as printed by \texttt{busybox} if the \texttt{--help} parameter is
given is embedded in the binary. The applet names are alphabetically sorted
and separated by NUL characters.

By searching for this list and splitting it accordingly it is possible to get
the list of all applets that are defined. The only caveats are that a new
applet that was added appears alphabetically before any of the applets that
can be recognized using a list of applet names extracted from the source code,
or it appears alphabetically after the last one that can be recognized.

\subsection{Pretty printing a BusyBox configuration}

Pretty printing a BusyBox configuration is fairly straightforward, but there
are a few cases where it is hard to make a good guess:

\begin{enumerate}
\item aliases
\item functionality that is added to an applet, depending on a configuration
directive
\item applets that use non-standard configuration names (like
\texttt{CONFIG\_APP\_UDHCPD} instead of \texttt{CONFIG\_UDHCPD} in some
versions of BusyBox)
\item features
\end{enumerate}

For some applets aliases are installed by default as symlinks. These aliases
are recorded in the binary, but there is no separate applet for it. In the
BusyBox sources (1.15.2, others might be different) these are defined as:

\begin{verbatim}
IF_CRYPTPW(APPLET_ODDNAME(mkpasswd, cryptpw, _BB_DIR_USR_BIN,
    _BB_SUID_DROP, mkpasswd))
\end{verbatim}

So if the \texttt{cryptw} tool is built, an additional symlink called
\texttt{mkpasswd} is added during installation.

If extra functionality is added to an applet in BusyBox it is defined in the
source code by macros like the following:

\begin{verbatim}
IF_SHA256SUM(APPLET_ODDNAME(sha256sum, md5_sha1_sum, _BB_DIR_USR_BIN,
    _BB_SUID_DROP, sha256sum))
IF_SHA512SUM(APPLET_ODDNAME(sha512sum, md5_sha1_sum, _BB_DIR_USR_BIN,
    _BB_SUID_DROP, sha512sum))
\end{verbatim}

The above configuration tells to add extra symlinks for \texttt{sha256sum} and
\texttt{sha512sum} if BusyBox is configured for suppport for the SHA256 and
SHA512 algorithms. The applet that implements this functionality is
\texttt{md5\_sha1\_sum}.

Non-standard configuration names can be fixed by using a translation table that
translates to the non-standard name. The current code has a translation table
for BusyBox 1.15 and higher.

Detecting features is really hard to do in a generic way. In most cases it will
even be impossible, because there are no clear markers (strings, applet names)
in the binary that indicate that a certain feature is enabled. In cases there
are clear marker strings these would still need to be linked to specific
features. One possibility would be to parse the BusyBox sources and link
strings to features, for example (from BusyBox 1.15.3,
\texttt{editors/diff.c}):

\begin{verbatim}
#if ENABLE_FEATURE_DIFF_DIR
       diffdir(f1, f2);
       return exit_status;
#else
       bb_error_msg_and_die("no support for directory comparison");
#endif
\end{verbatim}

The string \texttt{"no support for directory comparison"} only appears if the
feature \texttt{ENABLE\_FEATURE\_DIFF\_DIR} is not enabled.

Implementing this will be a lot of work and it will likely not be very useful.

\subsection{Using BusyBox configurations}

By referencing with information extracted from the standard BusyBox sourcecode
it is possible to get a far more accurate configuration, because it is known
which applets use which configuration, unless:

\begin{itemize}
\item new applets were added to BusyBox
\item applets use old names, but contain different code
\end{itemize}

The names of applets that are defined in BusyBox serve as a very good starting
point. How these are recorded in the sources has changed a few times and
depends on the version of BusyBox. The tool \texttt{appletname-extractor.py}
can extract these from the BusyBox sources and store them for later reference
as a simple lookup table in Python pickle format.

Names of applets per version breakdown:

\begin{itemize}
\item 1.15.x and later: \texttt{include/applets.h} or
\texttt{include/applets.src.h} IF syntax
\item 1.1.1-1.14.x: \texttt{include/applets.h} USE syntax
\item 1.00-1.1.0: \texttt{include/applets.h} (different syntax)
\item 0.60.5 and earlier: \texttt{applets.h}, like 1.00-1.1.0 but with a
slightly different syntax
\end{itemize}

In one particular version of BusyBox (namely 1.1.0) there is a mix of three
different syntaxes: (0.60.5, 1.00 and another) for a few applets
(\texttt{runlevel}, \texttt{watchdog}, \texttt{tr}).

There are also a few applets in 1.1.0 which seem to be a bit harder to detect:
\texttt{busybox}, \texttt{mkfs.ext3}, \texttt{e3fsck} and \texttt{[[}. These
can easily be added by hand, since there are just four of them.

Another issue that is currently unresolved is that not all the shells are
correctly recognized.

\subsection{Extracting configurations from BusyBox sourcecode}

The \texttt{busybox.py} script makes use of a table that maps applet names to
configuration directives. These tables are stored in a Python pickle and read
by \texttt{busybox.py} upon startup. To generate these pickle files the
\texttt{appletname-extractor.py} should be used. In the standard distribution
for BAT the configurations for most versions of BusyBox are shipped.

The applet names are extracted from a file called \texttt{applets.h} or
\texttt{applets.src.h}.

\begin{verbatim}
python appletname-extractor.py -a /path/to/applets.h -n $VERSION
\end{verbatim}

The configuration will be written to a file \texttt{\$VERSION-config} and
should be moved into the directory containing the other configurations.

\section{Linux kernel identifier extraction}

The \texttt{createdb.py} program processes Linux kernel source code files in a
slightly different way than normal source code files. There is a lot of
interesting information that can be extracted from the Linux kernel sources,
as well as the binary.

There are a few challenges when working with Linux kernel source code and Linux
kernel binaries. First of all there are many different variants in use and many
vendors have their own slightly modified version, with extra drivers, or bug
fixes from later versions, or bug fixes that might not yet have been applied to
the version on \url{kernel.org}.

Second is that in the Linux kernel binary string constants, function names,
symbols, module parameters, and so on, are intertwined and some steps need to
be taken to correctly split these to avoid false positives (there are other
packages where kernel function names, module parameters, symbols, and so on,
are valid string constants).

\subsection{Extracting visible strings from the Linux kernel binary}

If a kernel is an ELF binary (sometimes) the relevant sections of the
binary can be read using \texttt{readelf}. Otherwise \texttt{strings} can be
run on the binary. This method will return more strings than if using
\texttt{readelf}, but the extra strings are mostly extra cruft that have a low
chance of matching.

\subsection{Extracting visible strings from a Linux kernel module}

If a kernel module is an ELF binary (most cases) the relevant sections of the
binary can be read using \texttt{readelf}. Otherwise \texttt{strings} can be
run on the binary. This method will return more strings than if using
\texttt{readelf}, but the extra strings are mostly extra cruft that have a low
chance of matching.

\subsection{Extracting strings from the Linux kernel sources}

The Linux kernel is full of strings that can end up in a binary. Some
programmers have defined macros just specific to their part of the kernel for
ease of use (often a wrapper around \texttt{printk}, other programmers use
more standard mechanisms like \texttt{printk}. Most strings can be extracted
from the Linux kernel using \texttt{xgettext}. A minority of strings needs to
be extracted using a custom regular expression.

The following two cases are worth a closer look:

\subsubsection{\texttt{EXPORT\_SYMBOL} and \texttt{EXPORT\_SYMBOL\_GPL}}

The symbols defined in the \texttt{EXPORT\_SYMBOL} and
\texttt{EXPORT\_SYMBOL\_GPL} macros end up in the kernel image. The
\texttt{EXPORT\_SYMBOL\_GPL} symbol could be interesting for licensing
reporting as well, since anything that uses this symbol should be released
under the GPLv2. This is a topic for future research.

\subsubsection{\texttt{module\_param}}

The names of parameters for kernel modules can end up in the kernel, or in the
kernel module itself. The names of these parameters are typically prefixed
with the name of the module (which is often, but not always) and a dot, but
without the extension of the file. In cases where the module name does not
match the name of the file it was defined in extra information from the
build system needs to be added to determine the right string.

The code for this is in the function \texttt{\_\_init param\_sysfs\_builtin} in
\texttt{kernel/params.c}.

Module names are extracted from the kernel Makefiles and stored in the database
together with module information (author, license, description, parameters,
and so on).

\subsection{Forward porting and back porting}

There are some strings we scan for which might not be present in certain
versions, because they were removed, or not yet included in the mainline
kernel. A good example is devfs. This subsystem was removed in Linux kernel
2.6.17, but it is not safe to assume that this was done for every 2.6.17 (or
later) kernel that is out in the wild, since some vendors might have kept it
and ported it to newer versions (forward porting). Similarly code from newer
kernels might have been included in older versions
(backporting).

\subsection{Corner cases}

Sometimes a \texttt{\#define} or some configuration directive causes that our 
string matching method will not work, because the string is prepended with
extra characters.

An example from \texttt{arch/arm/mach-sa1100/dma.c} from kernel 2.6.32.9:

\begin{verbatim}
#undef DEBUG
#ifdef DEBUG
#define DPRINTK( s, arg... )  printk( "dma<%p>: " s, regs , ##arg )
#else
#define DPRINTK( x... )
#endif
\end{verbatim}

Other examples include \texttt{pr\_debug}, \texttt{DBG}, \texttt{DPRINTK} and
\texttt{pr\_info}.

To work around this there are two ways:

\begin{enumerate}
\item do substring matches
\item parse the source code and record where extra code is being added as in
the example above and only do substring matches in a small number of cases.
\end{enumerate}

Substring matching is expensive and since it only happens in a minority of
cases the second method, although not trivial to implement, would be easier.
This is future work.

\section{Binary Analysis Tool performance tips}

This section describes a few methods to increase performance of the Binary
Analysis Tool, plus describe drawbacks of methods named. The standard
configuration of BAT tries to be sensible, with a trade off between performance
and completeness. In some cases there is quite a bit of performance to be
gained by simply tweaking the configuration.

\subsection{Choose the right hardware}

BAT will benefit a lot from fast disk, enough memory and multiple cores. Many
of the scans in BAT can be run in parallel and will scale very well (until
of course disk I/O limits are reached). Invest in SSD to reduce disk I/O and
more cores instead of a faster CPU. Enough memory will prevent swapping which
just kills performance, especially because the ranking scan in BAT can be very
I/O intensive.

\subsection{Use \texttt{outputlite}}

Using the default configuration the original unpacked data is not included into
the result archive.

There are situations where it makes sense to include the data into the result
archive, for example to make it easier to do a ``post mortem'' after a scan.
The original data can take up a lot of space, since every original file, plus
everything that might have been extracted from that file, will be included,
which leads to large archives and long associated packing time.

It also has performance impact on the BAT viewer, which needs to unpack some
data from the archive. The smaller the archive is, the faster unpacking is.

If the original data and the unpacked data is not relevant, then setting the
option \texttt{outputlite} to \texttt{yes} in the section \texttt{[batconfig]}
is highly recommended:

\begin{verbatim}
outputlite = yes
\end{verbatim}

\subsection{Do not output results in XML}

By default BAT will output the results of a scan in XML. The information in
here more or less matches the information that is packed in the report. If the
XML file is not used for analysing resuts disabling pretty printing of the
results as XML can save time, especially if there are many scanned files with
many results.

Disabling the XML pretty printing can be disabled my outcommenting two
directives: \texttt{module} and \texttt{output}. In the default configuration
they have the following values:

\begin{verbatim}
module = bat.simpleprettyprint
output = prettyprintresxml
\end{verbatim}

\subsection{Use \texttt{AGGREGATE\_CLEAN} when scanning Java JAR files}

If Java JAR files are scanned then pictures and reports will be generated for
each of the individual \texttt{.class} files. If only the results of the JAR
file are needed, then setting \texttt{AGGREGATE\_CLEAN} to \texttt{1} will
prevent pictures and reports to be generated for the individual \texttt{.class}
files, which can save quite some processing time and help declutter the
interface as well.

Of course, not generating the pictures for individual \texttt{.class} files
means that some detail might be lost, especially if there are \texttt{.class}
files that contain some unexpected results.

\subsection{Disable \texttt{tmp} on tmpfs}

Some Linux distributions (most notably Fedora 18 and later) store the
\texttt{/tmp} file system on tmpfs. This means that part of the system memory
is used for the \texttt{/tmp} file system. By default on Fedora it is
set to 50\% of the system's memory. This could influence BAT in two ways:

\begin{enumerate}
\item less memory available for processing
\item BAT unpacks to \texttt{/tmp} by default, unless configured differently. If
the unpack results grow big enough (which is fairly easy with big firmwares) it
could fill up the partition. However, there are some external tools that will
write temporary results to \texttt{/tmp}.
\end{enumerate}

There are various solutions, apart from adding more memory to the machine:

\begin{itemize}
\item configure BAT to use another path than \texttt{/tmp} for unpacking and
storing results and configure some scans in BAT to use \texttt{/tmp} or a
different ramdisk (recommended)
\item disable \texttt{tmp} on tmpfs (not recommended)
\end{itemize}

\subsection{Use tmpfs for writing temporary results}

A few scans can use tmpfs or a ramdisk to write temporary results. The scans
that can benefit from this are LZMA unpacking, ranking (temporary results of
DEX and ODEX unpacking), compress unpacking, JFFS2 unpacking and TAR unpacking.

\section{Parameter description for default scans}

This section describes the default parameters for several of the scans as
shipped in BAT, if not described earlier in this document. These parameters are
passed to the scans as part of the environment and are defined in the
\texttt{envvars} setting in the configuration file.

\subsection{\texttt{jffs2}}

The \texttt{JFFS2\_TMPDIR} parameter is used to let the scan use a different
location for unpacking temporary files than the standard unpacking directory.
It was introduced to let the scan unpack onto a tmpfs file system to avoid disk
I/O and speed up scanning.

\subsection{\texttt{lzma}}

The \texttt{lzma} unpack scan has two parameters: \texttt{LZMA\_MINIMUM\_SIZE}
and \texttt{LZMA\_TMPDIR}.

The \texttt{LZMA\_MINIMUM\_SIZE} parameter instructs the scan to ignore output
files that are \texttt{LZMA\_MINIMUM\_SIZE} bytes in size or less. This
parameter was introduced because false positives in LZMA unpacking are very
common, often leading to small sized files that contain no useful data.

By default \texttt{LZMA\_MINIMUM\_SIZE} is set to 10 bytes, but this is a very
conservative setting.

The \texttt{LZMA\_TMPDIR} parameter is used to let the scan use a different
location for unpacking temporary files than the standard unpacking directory.
It was introduced to let the scan unpack onto a tmpfs file system to avoid disk
I/O and speed up scanning.

\subsection{\texttt{tar}}

The \texttt{TAR\_TMPDIR} parameter is used to let the scan use a different
location for unpacking temporary files than the standard unpacking directory.
It was introduced to let the scan unpack onto a tmpfs file system to avoid disk
I/O and speed up scanning.

\subsection{\texttt{file2package}}

The \texttt{file2package} leaf scan has one parameter:
\texttt{BAT\_PACKAGE\_DB}.  This parameter is used to specify the location of
the database used by this scan. The database can be generated using the scripts
\texttt{createfiledatabasedebian.py} and \texttt{createfiledatabasefedora.py}
in the subdirectory \texttt{maintenance} in the BAT source tree.

\subsection{\texttt{findlibs}}

For the \texttt{findlibs} aggregate scan the \texttt{ELF\_SVG} parameter can be
set to \texttt{1} to output the graphs in SVG format.

\subsection{\texttt{generateimages}}

The \texttt{generateimages} postrun scan has five optional parameters:
\texttt{AGGREGATE\_IMAGE\_SYMLINK}, \texttt{BAT\_IMAGEDIR}, \texttt{BAT\_PICKLEDIR}, \texttt{MAXIMUM\_PERCENTAGE}
\texttt{MINIMUM\_PERCENTAGE}

\subsection{\texttt{identifier}}

The \texttt{identifier} leaf scan has several parameters, including
\texttt{DEX\_TMPDIR}. This parameter can be used to set a location where
temporary files for DEX and ODEX (Android Dalvik files) unpacking can be
written. This would typically be tmpfs or a ramdisk.

\subsection{\texttt{licenseversion}}

The \texttt{licenseversion} aggregate scan has a few parameters that can
influence performance. One of them is \texttt{AGGREGATE\_CLEAN}. This parameter
instructs the scan to remove results for individual Java class files from the
result set after aggregating results at the JAR level. Java class files that
are not unpacked from a JAR file are not removed from the result set. By
default this parameter is set to 0 which means that results for Java class
files are not removed from the result set.

\subsection{\texttt{prunefiles}}

The \texttt{prunefiles} aggregate scan has two parameters: \texttt{PRUNE\_TAGS}
and \texttt{PRUNE\_FILEREPORT\_CLEAN}. The \texttt{PRUNE\_TAGS} parameter
contains a comma-separated list of tags that should be ignored and removed from
the scan results. The \texttt{PRUNE\_FILEREPORT\_CLEAN} parameter can be set to
indicate whether or not the result pickles for the pruned files should also be
removed from disk. Example:

\begin{verbatim}
PRUNE_TAGS=png,gif:PRUNE_FILEREPORT_CLEAN=1
\end{verbatim}

\subsection{\texttt{hexdump} and \texttt{images}}

The \texttt{hexdump} and \texttt{images} scans (disabled by default) have two
parameters. The \texttt{BAT\_IMAGE\_MAXFILESIZE} parameter is set to specify
the maximum size of a file for which a result is generated. Since output from
this scan can be extremely large, and the results are not very interesting for
large files it is strongly advised to cap this value.

\section{Default ordering of scans in BAT}

BAT comes with a default configuration file. In this file an order for running
the scans is specified, using the \texttt{priority} field: the higher the
priority, the earlier the scan is run in the process. In this section the
rationale behind this ordering is explained.

The order for pre-run scans, leaf scans, unpack scans and aggregate scans is
described below. Since postrun scans do not change the result files and they are
independent there is no order defined for them (although this might change in
the future).

\subsection{Pre-run scans}

Most pre-run scans have the same priority, with a few exceptions, the most
important being \texttt{verifytext} to find out if a file is ASCII only, or if
there are any non-ASCII characters in the file. Since many of the scans
(including pre-run scans) only work on non-ASCII files it is important to find
out soon if a file contains only ASCII characters or not.

The order for pre-run scans is:

\begin{enumerate}
\item \texttt{checkXML}
\item \texttt{verifytext}
\item \texttt{verifyjava}
\item \texttt{verifyelf}, \texttt{verifygraphics}, \texttt{verifysqlite3}
\item \texttt{verifyandroiddex}, \texttt{verifyandroidodex},
\texttt{verifyandroidresource}, \texttt{verifyandroidxml}, \texttt{verifybz2},
\texttt{verifygzip}, \texttt{verifyico}, \texttt{verifyjar},
\texttt{verifymessagecatalog}, \texttt{verifymp4}, \texttt{verifyogg},
\texttt{verifyotf}, \texttt{verifyttf}, \texttt{verifytz}, \texttt{verifywoff},
\texttt{vimswap}
\end{enumerate}

\subsection{Unpack scans}

As a general rule of thumb: compressed formats are scanned last, while
simple containers that concatenate contents, or where the original content can
still be (partially) recognised, are scanned first.

An example of a container is TAR: content is simply concatenated without
compression. If the TAR archive would contain a file of a certain type (such as
a gzip compressed file) and the unpacker for that type is run first it will try
to carve it from the TAR file, blacklist the byte range, and the TAR unpacker
would not successfully run.

For the compressed files on the other hand the original content isn't visible
without unpacking so no other scans will pick it up and they can have a low
priority.

The order that is defined starts with \texttt{byteSwap}, a special unpacker
that is needed to unpack firmwares of certain devices, where a different kind
of flash chip is used, needing bytes in a firmware to be swapped first before
any other scan can be run.

Then the unpack scans for various container formats and file systems are run.
The order in which they appear is not fool proof: container files could be
embedded in container files with a lower priority, but BAT comes with
(hopefully) sane defaults to prevent this.

Second to last unpack scans for compressed files where all data is packed in
such a way that the original content can't be seen without unpacking are run,

Finally there are some scans that unpack text files (\texttt{base64}) or media
files. The \texttt{lzma} unpack scan also has the lowest priority because of
possibly many false positives.

The order of the unpack scans as defined in BAT 19 is:

\begin{enumerate}
\item \texttt{byteSwap}
\item \texttt{tar}
\item \texttt{pdf\_unpack}, \texttt{iso9660}
\item \texttt{cramfs}, \texttt{ext2fs}
\item \texttt{ar}, \texttt{cpio}, \texttt{java\_serialized}, \texttt{romfs},
\texttt{rpm}, \texttt{upx}, \texttt{yaffs}
\item \texttt{exe}, \texttt{jffs2}, \texttt{squashfs}
\item \texttt{7z}, \texttt{arj}, \texttt{bzip2}, \texttt{cab},
\texttt{compress}, \texttt{gzip}, \texttt{installshield}, \texttt{lrzip},
\texttt{lzip}, \texttt{lzo}, \texttt{pack200}, \texttt{rar}, \texttt{rzip},
\texttt{xz}, \texttt{zip}
\item \texttt{base64}, \texttt{gif}, \texttt{ico}, \texttt{png}, \texttt{swf},
\texttt{lzma}
\end{enumerate}

\subsection{Leaf scans}

There is currently only one explicit ordering: \texttt{kernelchecks} is run
before \texttt{identifier} because \texttt{identifier} depends on the result
of \texttt{kernelchecks}. For the rest the order of the leaf scans does not
matter.

\subsection{Aggregate scans}

Aggregate scans have a clear order. Reports and (most) images are generated at
the very end when all information is known. Other scans are mostly independent
of eachother, but are run before \texttt{versionlicensecopyright} to prevent
having to read big report pickles from disk.

The order for pre-run scans is:

\begin{enumerate}
\item \texttt{prunefiles} (disabled by default)
\item \texttt{findduplicates}
\item \texttt{findlibs}
\item \texttt{jars}, \texttt{kernelversions}
\item \texttt{versionlicensecopyright}
\item \texttt{generateimages}, \texttt{generatereports}
\end{enumerate}

\end{document}
