\documentclass[11pt]{beamer}

\usepackage{url}
\usepackage{tikz}
%\author{Armijn Hemel}
\title{Using the Binary Analysis Tool - part 3}
\date{}

\begin{document}

\setlength{\parskip}{4pt}

\frame{\titlepage}

\frame{
\frametitle{Subjects}
In this course you will learn:

\begin{itemize}
\item scan order in the Binary Analysis Tool
\item to configure the Binary Analysis Tool
\item to browse results of a scan made with the Binary Analysis Tool
\end{itemize}
}

\frame{
\frametitle{Scan order in the Binary Analsysis Tool}

\begin{enumerate}
\item the binary is read and file type specific identifiers (from a hardcoded list) are searched for in the binary
\item prerun scans are run to tag files to filter out specific file types later
\item unpacking scans are run to unpack any compressed files or file systems and scans 1 - 3 are run recursively
\item file specific scans are run on all unpacked files
\item post run scans are run for each individual file
\end{enumerate}
}

\frame{
\frametitle{Identifier search}
The identifier search uses a hardcoded list of identifiers that indicate where a certain file starts or stops. Not all file types have identifiers, but most do.

The locations of identifiers are passed to later scans, which can use this information to work in a more efficient way.
}

\frame{
\frametitle{Prerun scans}
Prerun scans are run to determine the type of the \textit{entire} file and to tag it. Tags can be used by later scans to ignore files: a scan to unpack a file system won't work on a graphics file or a text file.

Some tags that can be set by prerun scans as shipped by BAT are:

\begin{itemize}
\item text
\item binary
\item graphics
\item audio
\item elf
\end{itemize}
}

\frame{
\frametitle{Unpacking scans}

Unpacking scans try to extract file systems and compressed files, sometimes from a larger binary blob, by ``carving'' it from a larger binary. Currently some 30 file systems and compressed files are supported:

\begin{itemize}
\item file systems: cramfs, ext2/ext3/ext4, ISO9660, JFFS2, Minix (specific
variant of v1), SquashFS (several variants), romfs, YAFFS2 (specific variants)
\item compressed files and executable formats: 7z, ar, ARJ, BASE64, BZIP2,
compressed Flash, CAB, compress, CPIO, EXE (specific compression methods only),
GZIP, InstallShield (old versions), LRZIP, LZIP, LZMA, LZO, RAR, RPM, serialized
Java, TAR, UPX, XZ, ZIP
\item media files: GIF, ICO, PDF, PNG
\end{itemize}
}

\frame{
\frametitle{File specific scans}
}

\frame{
\frametitle{Postrun scans}
Postrun scans are scans that don't modify the files (unpacking, or scanning files), but merely act on results of scans, for example:

\begin{itemize}
\item generating reports
\item generating pictures
\end{itemize}
}

\frame{
\frametitle{Configuring the Binary Analysis Tool}

The Binary Analysis Tool is highly configurable and uses plugins. These plugins can be enabled and disabled via a configuration file.

The configuration file is in Windows INI format and contains several parts:

\begin{itemize}
\item general configuration
\item configuration directives for ``prerun'' scans
\item configuration directives for ``unpack'' scans
\item configuration directives for per file scans
\end{itemize}

The general configuration is mandatory, the other directives are optional.
}

\begin{frame}[fragile]
\frametitle{General configuration}

\begin{verbatim}
[batconfig]
multiprocessing = no
module = bat.simpleprettyprint
output = prettyprintresxml
\end{verbatim}

The \texttt{multiprocessing} option is to enable the use of multiple processors. Usually this is safe, unless one of the scans is not safe (for example because it writes to a database).

%The \texttt{output} and \texttt{module} options are used together to 

\end{frame}

\frame{
\frametitle{Mandatory scan configuration options}
The configuration for each type of scan has a few mandatory options:

\begin{itemize}
\item \texttt{type} - \texttt{prerun}, \texttt{unpack}, \texttt{program} or \texttt{postrun}
\item \texttt{module} - Python module (including package) the scan can be found
\item \texttt{method} - the method for the scan
\item \texttt{enabled} - \texttt{yes} enables a scan, \texttt{no} disables a scan
\end{itemize}
}

\frame{
\frametitle{Optional scan configuration options}

\begin{itemize}
\item \texttt{priority}
\item \texttt{noscan}
\item \texttt{magic}
\item \texttt{description}
\end{itemize}
}

\begin{frame}[fragile]
\frametitle{Prerun scans configuration directive example}

\begin{verbatim}
[checkXML]
type        = prerun
module      = bat.prerun
method      = searchXML
priority    = 100
description = Check XML validity
enabled     = yes
\end{verbatim}

\end{frame}

\begin{frame}[fragile]
\frametitle{Unpacking scans configuration directive example}

The following example is for unpacking of 7z archives. The priority is rather low, only the identifier for \texttt{7z} is used, and a bunch of file types can safely be ignored by this scan:

\begin{verbatim}
[7z]
type        = unpack
module      = bat.fwunpack
method      = searchUnpack7z
priority    = 1
magic       = 7z
noscan      = text:xml:graphics:pdf:bz2:gzip:lrzip:audio:video
description = Unpack 7z compressed files
enabled     = yes
\end{verbatim}

\end{frame}

\frame{
\frametitle{Examining the results of the Binary Analysis Tool}
The output of the Binary Analysis Tool is written as a tar archive. The tar archive consists of:

\begin{itemize}
\item full directory tree of unpacked files (if any)
\item Python pickles with the results of the scan
\item (optional) pictures with results of the scan
\end{itemize}

The results can be viewed using the Binary Analysis Tool result viewer.
}

\frame{
\frametitle{Starting the Binary Analysis Tool result viewer}
The Binary Analysis Tool result viewer is a Python program using wxPython. It can be invoked using the command:

\texttt{batgui}

which will launch the GUI.

For some functionality the BAT configuration file is needed. The configuration file can be loaded from the GUI, or supplied on the commandline:

\texttt{batgui -c /path/to/configuration/file}
}

\begin{frame}[fragile]
\frametitle{Configuring the Binary Analysis Tool result viewer}
\end{frame}

\frame{
\frametitle{Loading a file in the BAT result viewer}

Via File $\rightarrow$ Open in the menu a result file can be loaded and will be unpacked.

On the left there will be a file tree, on the right results will be displayed.
}

\frame{
\frametitle{Filtering results in the BAT result viewer}
Not every file type might be interesting. To unclutter the user interface and the directory tree a display filter is present that will hide certain file types from the directory tree.

Configuration $\rightarrow$ Filter Configuration will show a list of checkboxes of file types to ignore.
}

\frame{
\frametitle{Interpreting results of a scan}
For each file a few attributes will be shown by default:

\begin{itemize}
\item name of the binary
\item absolute file path
\item relative file path if it is nested and parent is an unpacked compressed file or file system
\item size
\item SHA256 checksum
\item tags (as determined by pre run scans)
\end{itemize}

In addition results of file specific scans might be shown (architecture, shared libraries, etcetera)
}

\frame{
\frametitle{Interpreting results of advanced ranking scan}
If the advanced ranking scan is enabled a lot more information becomes available:

\begin{itemize}
\item function names matching
\item string constants matching
\item version number guess
\end{itemize}
}

\frame{
\frametitle{Interpreting results: function names}
}

\frame{
\frametitle{Interpreting results: string constants}
}

\frame{
\frametitle{Interpreting results: version numbers}
}

\frame{
\frametitle{Advanced mode in BAT result viewer}
The BAT result viewer also has an ``advanced mode''. Advanced mode can be enabled via:

Configuration $\rightarrow$ General Configuration $\rightarrow$ Advanced mode

When enabled a tab ``Alternate view'' will appear.

When during scanning the scans \texttt{hexdump} and \texttt{images} (available in default BAT distribution) were enabled, different representations of files have been generated:

\begin{itemize}
\item output of \texttt{hexdump -C}
\item picture where every byte has been replaced by a grayscale pixel
\end{itemize}

These two representations are correlated: clicking on the picture will display the corresponding section in the \texttt{hexdump} output.

This is not enabled by default since it is quite resource intensive.
}

\frame{
\frametitle{Scanning via the BAT result viewer}
}
\end{document}
