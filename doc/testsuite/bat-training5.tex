\documentclass[11pt]{beamer}

\usepackage{url}
\usepackage{tikz}
%\author{Armijn Hemel}
\title{Using the Binary Analysis Tool - part 5}
\date{}

\begin{document}

\setlength{\parskip}{4pt}

\frame{\titlepage}

\frame{
\frametitle{Subjects}
In this course you will learn:

\begin{itemize}
\item adding new identifiers to BAT
\item writing a new prerun scan
\item writing a new unpack scan
\item writing a new file specific scan
\item writing a new postrun scan
\end{itemize}
}

\frame{
\frametitle{Adding identifiers to identifier search}
Identifiers for files are stored in \texttt{bat/fsmagic.py}. To add an identifier:

\begin{enumerate}
\item add it to the \texttt{fsmagic} dictionary
\item add an offset to the \texttt{correction} dictionary if the identifier does not start at byte 0
\item optionally group identifiers in an array, similar to \texttt{squashtypes}
\end{enumerate}

A new identifier will be scanned as soon as there is a scan that has declared it in the configuration in the \texttt{magic} option for the scan.
}

\begin{frame}[fragile]
\frametitle{Using identifiers in scans}
Identifiers can be accessed in a dictionary called \texttt{offsets}. This dictionary is passed around as an argument
 to each scan and can be accessed by the keys from the \texttt{fsmagic} dictionary, for example:

\begin{verbatim}
if offsets['java_serialized'] == []:
        return ([], blacklist, [])
\end{verbatim}
\end{frame}

\frame{
\frametitle{Writing a new prerun scan}
A prerun scan takes 5 parameters:

\begin{itemize}
\item \texttt{filename} - the full filename of the file during scanning
\item \texttt{tempdir} (default \texttt{None}) - the temporary directory where the file has been unpacked
\item \texttt{tags} (default \texttt{[]}) - any existing tags for this file (from other prerun scans)
\item \texttt{offsets} (default \texttt{\{\}}) - offsets of identifiers
\item \texttt{envvars} (default \texttt{None}) - optional environment variables
\end{itemize}
The return value should be a list of tags (strings), or an empty list.
}

\begin{frame}[fragile]
\frametitle{Prerun scan example}
\begin{verbatim}
def searchXML(filename, tempdir=None, tags=[],
              offsets={}, envvars=None):

    newtags = []

    p = subprocess.Popen(['xmllint','--noout',
        "--nonet", filename], stdout=subprocess.PIPE,
        stderr=subprocess.PIPE, close_fds=True)

    (stanout, stanerr) = p.communicate()

    if p.returncode == 0:
        newtags.append("xml")

    return newtags
\end{verbatim}
\end{frame}

\frame{
\frametitle{Writing a new unpacking scan}
\begin{itemize}
\item \texttt{filename} - the full filename of the file during scanning
\item \texttt{tempdir} (default \texttt{None}) - the temporary directory where the file has been unpacked
\item \texttt{blacklist} (default \texttt{[]}) -
\item \texttt{offsets} (default \texttt{\{\}}) - offsets of identifiers
\item \texttt{envvars} (default \texttt{None}) - optional environment variables
\end{itemize}
}

\frame{
\frametitle{Writing a new file scan}

\begin{itemize}
\item \texttt{path} - the full filename of the file during scanning
\item \texttt{blacklist} (default \texttt{[]}) -
\item \texttt{envvars} (default \texttt{None}) - optional environment variables
\end{itemize}
}

\begin{frame}[fragile]
\frametitle{File scan example}
\begin{verbatim}
def scanVirus(path, blacklist=[], envvars=None):

  p = subprocess.Popen(['clamscan', "%s" % (path,)],
      stdout=subprocess.PIPE, stderr=subprocess.PIPE,
      close_fds=True)

  (stanout, stanerr) = p.communicate()

  if p.returncode == 0:
    return
  else:
    # virus found
    viruslines = stanout.split("\n")
    virusname = viruslines[0].strip()[len(path) + 2:-6]
    return virusname
\end{verbatim}
\end{frame}

\frame{
\frametitle{Writing a new postrun scan}

\begin{itemize}
\item \texttt{filename} - the full filename of the file during scanning
\item \texttt{unpackreport}
\item \texttt{leafscans}
\item \texttt{scantempdir}
\item \texttt{toplevelscandir}
\item \texttt{envvars} (default \texttt{None}) - optional environment variables
\end{itemize}

Since postrun scans are only supposed to have side effects no return value is expected.

Example: \texttt{bat/generatehexdump.py}
}

\frame{
\frametitle{Conclusion}
In this course you have learned about:

\begin{itemize}
\item adding new identifiers to BAT
\item writing a new prerun scan
\item writing a new unpack scan
\item writing a new file specific scan
\item writing a new postrun scan
\end{itemize}

In the next course we will show how a database for the Binary Analysis Tool can be created.
}

\end{document}
