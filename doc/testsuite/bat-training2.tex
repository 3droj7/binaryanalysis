\documentclass[11pt]{beamer}

\usepackage{url}
\usepackage{tikz}
%\author{Armijn Hemel}
\title{Using the Binary Analysis Tool - part 2}
%\date{January 17, 2011}
\date{}

\begin{document}

\setlength{\parskip}{4pt}

\frame{\titlepage}

\frame{
\frametitle{Subjects}

In this course you will learn:

\begin{itemize}
\item examining individual files for clues of which software is being used
\item automated scanning of individual files with the Binary Analysis Tool
\end{itemize}
}

\frame{
\frametitle{Before we start}
It is important to keep in mind that this is only for finding evidence. Whether or not someone violates a license is a legal question, not a technical one.
}

\frame{
\frametitle{Examining individual files}

\begin{itemize}
\item file type
\item dynamically linked libraries
\item extracting human readable strings
\item match strings extracted from binaries with source code
\item extracting function names from ELF executables
\item match function names extracted from binaries with source code
\end{itemize}
}

\frame{
\frametitle{Interesting files for compliance analysis}
There are various types of files we are interested in for license issues:

\begin{itemize}
\item executables (ELF, bFLT, Java, Windows executables)
\item libraries
\item Linux kernel images
\end{itemize}

Other files, such as audio, video or graphics files, can be interesting too for trademarks, or licensing (Creative Commons), but are outside of the scope of this course.
}

\frame{
\frametitle{Determining file types}
You can make a good first guess with the \texttt{file} command, which uses the magic database using \texttt{libmagic}.

\texttt{libmagic} uses the so called magic database which contains a description for many sorts of files. It is usually found in \texttt{/usr/share/magic} on Linux systems.
}

\frame{
\frametitle{Excercise: determine the type of files}

Use the \texttt{file} command to recognize various files on your system.
}

\frame{
\frametitle{Drawbacks of using \texttt{file} and \texttt{libmagic}}
There are some drawbacks to using commands like \texttt{file} or tools that use \texttt{libmagic}:

\begin{itemize}
\item the descriptions in the magic database can and will change over time and sometimes contains errors.
\item the magic database is not complete.
\item some file types don't have a fixed magic type.
\item not the entire file is considered, but just a number of bytes (up to a few hundred), so false positives can happen.
\end{itemize}

In BAT data from \texttt{libmagic} is used for reporting, but not for searching and unpacking, only for a few occassional verification steps.
}

\frame{
\frametitle{File type verification in BAT}
In BAT a \textit{conservative} approach is used:

\begin{itemize}
\item use standard tools to verify, even though sometimes it costs more time, for example \texttt{gifinfo} for GIF files. These tools often take the whole file into account.
\item err on the safe side: if not entirely sure (or sure enough) then don't assume a file is of a certain type
\end{itemize}
}

\frame{
\frametitle{Inspecting dynamically linked ELF files}
In case the file is an ELF executable (executable program) or shared object (library for dynamic linking) it is important to find the runtime dependencies.
}

\frame{
\frametitle{Dynamic linking 101}
ELF executables can be linked in two ways:

\begin{itemize}
\item static linking: all dependencies (C library, other libraries) are included into the final executable.
\item dynamic linking: dependencies are resolved at the moment the program is run.
\end{itemize}

For license requirements it is very significant to see what a program links to (either dynamically or statically). If a program is dynamically linked you need to inspect the file to find the dependencies.
}

\begin{frame}[fragile]
\frametitle{Using \texttt{readelf} to display shared libraries}
The best tool to display shared libraries is to use the \texttt{readelf}, which is part of GNU binutils.

The command:

\texttt{readelf -d | grep NEEDED}

will display the shared libraries that are used, for example (output formatted using \texttt{cut}):

\begin{verbatim}
$ readelf -d /bin/ls | grep NEEDED | cut -f 4- -d " "
            Shared library: [libselinux.so.1]
            Shared library: [librt.so.1]
            Shared library: [libcap.so.2]
            Shared library: [libacl.so.1]
            Shared library: [libc.so.6]
\end{verbatim}
\end{frame}

\frame{
\frametitle{Why not use \texttt{ldd}?}

One reason to not use \texttt{ldd} is that \texttt{ldd} uses the dynamic linker on the \textit{local} system to find dependencies.

It might seem that an executable or library is linked with a certain library, while in fact it is not and just a dependency on the \textit{local} system.

Also, it will not work for executables compiled for different architectures.
}

\frame{
\frametitle{Excercise: find the dynamically linked libraries of a file}
\begin{enumerate}
\item Look at several dynamically linked executables on your system with \texttt{readelf}. Also look at them with \texttt{ldd} and try to spot differences.
\item repeat, but with files from the test firmware(s)
\end{enumerate}
}

\frame{
\frametitle{Extracting human readable strings}
\texttt{strings}
}

\frame{
\frametitle{Excercise: }
}

\frame{
\frametitle{Matching strings from binaries with source code}
}

\frame{
\frametitle{Excercise: }
}

\frame{
\frametitle{Extracting function names from dynamically linked ELF files}
}

\end{document}
